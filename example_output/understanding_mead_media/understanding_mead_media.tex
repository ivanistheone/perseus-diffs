\documentclass[twocolumn,10pt]{article}
\title{Understanding mean and media}
\setlength{\columnsep}{20pt} 
\usepackage{amsmath,hyperref,cancel,graphicx}
 \def\shrinkfactor{0.55}
 \usepackage[margin=1.5cm]{geometry}
\usepackage[usenames,dvipsnames]{color}
 
 \newcommand{\blue}[1]{{\color{Blue}#1}} 
 \newcommand{\purple}[1]{{\color{Purple}#1}} 
 \newcommand{\red}[1]{{\color{Red}#1}} 
 \newcommand{\green}[1]{{\color{Green}#1}} 
 \newcommand{\gray}[1]{{\color{Gray}#1}} 
  \newcommand{\pink}[1]{{\color{Magenta}#1}}   


\begin{document}
\maketitle



\section{\href{https://www.khanacademy.org/devadmin/content/items/x12501ec401d7d33c}{x12501ec401d7d33c}}

\noindent
These are the scores for Sam's English tests: $78, ~80,~ 82,~ 85,~ 90$. 

The median is the middle score, i.e. $ 82$. 

**Choose the correct statement(s).**

\paragraph{Ans} 

\fbox{ Sam scored $82$ on one test

}

 \fbox{ Sam had as many test scores below $82$ as above

}

 The average of all of Sam's scores is $82$

Sam scored $82$ in all the tests



\paragraph{Hint 1}Depending on whether the number of scores is odd or even, the median score is the middle score or the average of the two middle scores.  When the number of scores is odd as in this case, the median score is equal to one of the test scores.  

\paragraph{Hint 2}However, the median need not be equal to the average score.  In this case, the average of the five scores is not equal to the median score.

\paragraph{Hint 3}"Sam had as many test scores below $82$ as above", and

 "Sam scored $82$ on one test"

are correct.



\medskip
\noindent
\textbf{Tags:} {\footnotesize CC.6.SP.A.2, SB.6.1.I.3.CR, Understanding mean and median}\\
\textbf{Version:} 37fa5355.. 2013-10-07
\smallskip\hrule





\section{\href{https://www.khanacademy.org/devadmin/content/items/x13a4544e8e3e0888}{x13a4544e8e3e0888}}

\noindent
Lisa collected the following information on number of bedrooms in houses in her neighborhood: 

$\qquad 2,~1,~3,~4,~2,~3,~3,~1,~4,~3,~2,~2,~2,~3$


**The mean number of bedrooms is [[? input-number 1]].**  
**The median number of bedrooms is [[? input-number 2]] .**

**Is the mean or the median a better choice for the measure of center of this data set?**

\paragraph{Ans} 

The mean is a better choice because the numbers are almost equally spread around the mean, but not around the median.

The median is a better choice because the mean is  high compared to most of the numbers.

\fbox{ 
\parbox{0.4\textwidth}{ 
The mean or the median could be used as the measure of center.
}

}

 

\paragraph{Hint 1}We can calculate the mean as follows:

\begin{align*}\text{Mean }&=\dfrac{\text{Sum of number of bedrooms in all the houses}}{\text{Number of houses}}\\
\\
&=\dfrac{2+1+3+4+2+3+3+1+4+3+2+2+2+3}{14}\\
\\
&= \dfrac{35}{14}\\
\\
&=2.5\end{align*}

\paragraph{Hint 2}In this case the median is the average of the two middle numbers when all of them are arranged in order:

$\qquad1,~1,~2,~2,~2,~2,\red{~2,~3},~3,~3,~3,~3,~4,~4$

\begin{align*}\text{median}&=\dfrac{2+3}{2}\\
\\
&=2.5\end{align*}

\paragraph{Hint 3}Since the mean and the median are equal to each other, any one could be used as the measure of center.

\paragraph{Hint 4}The mean as well as the median number of bedrooms is $2.5$.  The mean or the median could be used as the measure of center.



\medskip
\noindent
\textbf{Tags:} {\footnotesize CC.6.SP.A.3, SB.6.1.I.4.SR, Understanding mean and median}\\
\textbf{Version:} 5e896c18.. 2013-10-07
\smallskip\hrule





\section{\href{https://www.khanacademy.org/devadmin/content/items/x1ff165cef2f387d4}{x1ff165cef2f387d4}}

\noindent
There are $2, ~1, ~3, ~1,$ and $4$ children, respectively in $5$ families who live on the same street.  Therefore, the mean number of children in a family is $2.2$, i.e.

 $\qquad  \dfrac{(2+1+3+1+4)}{5} = 2.2$. 

**Choose the correct statement.**

\paragraph{Ans} 

Exactly one family has $2.2$ children.

There are as many families with less than $2.2$ children as those above.

\fbox{ The average number of children for the $5$ families is $2.2$

}

 All the $5$ families have $2.2$ children each.



\paragraph{Hint 1}The mean is the average, calculated in this case as the total number of children divided by the number of families.  The mean may or may not be equal to any of the individual values; and may or may not be equal to the middle value.

\paragraph{Hint 2}"The average number of children for the $5$ families is $2.2$" is the correct statement.



\medskip
\noindent
\textbf{Tags:} {\footnotesize CC.6.SP.A.2, SB.6.1.I.3.CR, Understanding mean and median}\\
\textbf{Version:} d5e5b3e1.. 2013-10-07
\smallskip\hrule





\section{\href{https://www.khanacademy.org/devadmin/content/items/x25848fe18ef6021a}{x25848fe18ef6021a}}

\noindent
During the regular football season $2012$, the points scored by the $39$ers in their games were as follows:

$\qquad30,~27,~13,~34,~45,~3,~13,~24,~24,~32,~31,~13,~27,~41,~13,~27$

**How many points did they score per game on average?**

\paragraph{Ans} 

\paragraph{Hint 1}We can calculate the average by dividing total points scored in all the games by the total number of games.

\paragraph{Hint 2}\begin{align*}\text{Average points per game} &= \dfrac{\text{Total points scored in all the games}}{\text{Number of games}}\\ 
\\
&= \dfrac{397}{16}\\
\\
&=24.8\end{align*}


\paragraph{Hint 3}The $39$ers scored $24.8$ points per game on average.



\medskip
\noindent
\textbf{Tags:} {\footnotesize CC.6.SP.A.2, SB.6.1.I.3.CR, Understanding mean and median}\\
\textbf{Version:} 5b70db23.. 2013-10-07
\smallskip\hrule





\section{\href{https://www.khanacademy.org/devadmin/content/items/x38d7f233055d08c4}{x38d7f233055d08c4}}

\noindent
During the regular football season $2012$, the points scored by the $39$ers in their games were as follows:

$\qquad30,~27,~13,~34,~45,~3,~13,~24,~24,~32,~31,~13,~27,~41,~13,~27$

**What was the team's most common score during the season?**

\paragraph{Ans} 

\paragraph{Hint 1}The $39$ers scored $13$ points most often.



\medskip
\noindent
\textbf{Tags:} {\footnotesize CC.6.SP.A.2, SB.6.1.I.3.CR, Understanding mean and median}\\
\textbf{Version:} c7de8775.. 2013-10-07
\smallskip\hrule





\section{\href{https://www.khanacademy.org/devadmin/content/items/x3942c6557a955c9d}{x3942c6557a955c9d}}

\noindent
Over the last $15$ days the maximum daily temperatures (in degrees Fahrenheit) in a city have been as follows:

$\qquad72,~72,~78,~74,~76,~72,~70,~66,~71,~73,~70,~72,~75,~74,~78$

The mean temperature is [[? input-number 1]] degrees Fahrenheit.

The median temperature is [[? input-number 2]] degrees Fahrenheit.

A group of tourists wants to know how warm it has been in the last $15$ days.

**Choose the correct statement.**


\paragraph{Ans} 

The tourists could be told about the mean temperature as it is more typical than the median.

The tourists could be told about the median temperature as it is more typical than the mean.

\fbox{ 
\parbox{0.4\textwidth}{ 
The tourists could be told about the mean or the 
median temperature as both are equally typical.
}
}

 

\paragraph{Hint 1}We calculate the arithmetic mean by dividing the total of all daily maximum temperatures by the number of days.

\paragraph{Hint 2}\begin{align*}\text{The arithmetic mean} &= \dfrac{\text{Total of all temperatures}}{\text{Number of days}}\\ 
\\
&= \dfrac{1093}{15}\\
\\
&=72.9\end{align*}

\paragraph{Hint 3}The median is the middle temperature when the $15$ temperatures are arranged in order.

$\qquad66,~70,~70,~71,~72,~72,~72,~\green{72},~73,~74,~74,~75,~76,~78,~78$

\paragraph{Hint 4}Since the mean and the median are almost the same, "the tourists could be told about the mean or the median as both are equally typical."



\medskip
\noindent
\textbf{Tags:} {\footnotesize CC.6.SP.A.2, SB.6.1.I.3.CR, Understanding mean and median}\\
\textbf{Version:} 52067da2.. 2013-10-09
\smallskip\hrule





\section{\href{https://www.khanacademy.org/devadmin/content/items/x3bc79e6496f3ef60}{x3bc79e6496f3ef60}}

\noindent
There are $1, 1, 2, 3,$ and $3$ children, respectively in 5 families who live on the same street.  Therefore, the mean number of children in a family is $2$, i.e.

 $\qquad \dfrac{(1+1+2+3+3)}{5} = 2$.

**Choose the correct statement(s).**


\paragraph{Ans} 

\fbox{ Exactly one family has $2$ children.

}

 \fbox{ 
 \parbox{0.4\textwidth}{ 
 There are as many families with less than $2$ children as those above.
 }

}

 \fbox{ The average number of children for the $5$ families is $2$

}

 All the $5$ families have $2$ children each.



\paragraph{Hint 1}The mean is the average, calculated in this case as the total number of children divided by the number of families.  

\paragraph{Hint 2}The mean may or may not be equal to any of the individual values (e.g. the mean of $2,3,$ and $7$ is $4$, which is not equal to any of the values), but in this case there is one family with 2 children.

\paragraph{Hint 3}The mean may or may not be the middle value (e.g. $2,3,$ and $7$; the mean is $4$ but the middle value is $3$), but in this case $2$ is also the middle of the $5$ numbers; and there are $2$ families with less than $2$ children and $2$ families with more. 

\paragraph{Hint 4}"The average number of children for the $5$ families is $2$",

"Exactly one family has $2$ children",

"There are as many families with less than $2$ children as those above" ,

are all correct statements.



\medskip
\noindent
\textbf{Tags:} {\footnotesize CC.6.SP.A.2, SB.6.1.I.3.CR, Understanding mean and median}\\
\textbf{Version:} f690cbe5.. 2013-10-07
\smallskip\hrule





\section{\href{https://www.khanacademy.org/devadmin/content/items/x50bd6292aacbd047}{x50bd6292aacbd047}}

\noindent
According to the census each household in a certain city has on average $3.85$ members.

**Choose the correct statement(s).**

\paragraph{Ans} 

Each household in the city has $3.85$ members.

There are some households in the city that have $3.85$ members.

The census is invalid because there is no household in the city with $3.85$ members.

\fbox{ Many households in the city are likely to have $3$ to $5$ members each.

}

 

\paragraph{Hint 1}It is impossible to have fractional number of members in any household because people come in whole numbers! Does the average also need to be a whole number?

\paragraph{Hint 2}The average number of members in a household is a ratio.

 $\text{Average number of members}=\dfrac{\text{Total number of people in all households}}{\text{number of households}}$

For example, there could be $385$ people living in $100$ households, giving us an average of $3.85$.

\paragraph{Hint 3}"Many households in the city are likely to have 3 to 5 members each." is the correct statement.



\medskip
\noindent
\textbf{Tags:} {\footnotesize CC.6.SP.A.2, SB.6.1.I.3.CR, Understanding mean and median}\\
\textbf{Version:} e48bb556.. 2013-10-07
\smallskip\hrule





\section{\href{https://www.khanacademy.org/devadmin/content/items/x58d4621135f229f1}{x58d4621135f229f1}}

\noindent
There are $3$ classes in a school with $32, ~35,$ and $ 41$ students, respectively. 

**How many students would there be in each class  if each class had the same number of students?**

\paragraph{Ans} [[? input-number 1]] students  36

\paragraph{Hint 1}We can use the average to have the same number of students in each of the classes.

\begin{align*}\text{Average} &= \dfrac{\text{Total number of studens in all the classes}}{\text{Number of classes}}\\
\\
&= \dfrac{32+35+41}{3}\\
\\&=36\end{align*}

\paragraph{Hint 2}There would be $36$ students in each class.



\medskip
\noindent
\textbf{Tags:} {\footnotesize CC.6.SP.A.2, SB.6.1.I.3.CR, Understanding mean and median}\\
\textbf{Version:} 154ef868.. 2013-10-07
\smallskip\hrule





\section{\href{https://www.khanacademy.org/devadmin/content/items/x5b321d412a594418}{x5b321d412a594418}}

\noindent
The geographical sizes (in thousands of square kms) of Post-Soviet countries are shown in the table.

**The average size is  [[? input-number 1]] thousand square kms.**  
**The median size is [[? input-number 2]] thousand square kms.**

**Is the mean or the median a better choice for the measure of center of this data set?**

Country | Size | 
- | - | -
Armenia | 12 | 
Azerbaijan| 87 | 
Belarus |208 | 
Estonia|45|
Georgia|68|
Kazakhstan| 2725|
 Kyrgyzstan| 200|
 Latvia| 65| 
Lithuania| 65|
 Moldova|34|
Russia| 17098| 
Tajikistan| 143| 
Turkmenistan| 5126|
 Ukraine|604|
 Uzbekistan|447



\paragraph{Ans} 

The mean is a better choice because the sizes are almost equally distributed around the mean sizes.

\fbox{
\parbox{0.4\textwidth}{ 
 The median is a better choice because the mean is  high compared to most of the sizes.
 }

}

 The mean or the median could be used as the measure of center because both are equally good.



\paragraph{Hint 1}We would calculate the mean size as follows:

\begin{align*}\text{Mean size}&=\dfrac{\text{Sum of areas of all the countries}}{\text{Number of countries}}\\
\\
&=\dfrac{26927}{15}\\
\\
&= 1795\end{align*}

\paragraph{Hint 2}The median is the middle number when the areas are arranged in order:

$\qquad12,~34,~45,~65,~68,~68,~87,\red{~143},~200,~208,~447,~604,~2725,~5126,~17098$

$\text{Median size}=143$

\paragraph{Hint 3}Only $3$ countries out of $15$ are greater in size than the mean size of $1795$ thousand square kms. Therefore, the median is a better choice as the measure of center.

\paragraph{Hint 4}The mean size $1795$ thousand square kms.  The median size is $ 143$ thousand square kms.  The median is a better choice because the mean is  high compared to most of the sizes.



\medskip
\noindent
\textbf{Tags:} {\footnotesize CC.6.SP.A.3, SB.6.1.I.4.SR, Understanding mean and median}\\
\textbf{Version:} ec8dd07e.. 2013-10-07
\smallskip\hrule





\section{\href{https://www.khanacademy.org/devadmin/content/items/x5d79f3f689e4505e}{x5d79f3f689e4505e}}

\noindent
John got the following scores on his science tests this year:

$\qquad80,~85,~85,~90,~85,~85,~95,~85,~80$

**What score did John get most frequently?**



\paragraph{Ans} 

\paragraph{Hint 1}John got a score of $85$ most frequently.



\medskip
\noindent
\textbf{Tags:} {\footnotesize CC.6.SP.A.2, SB.6.1.I.3.CR, Understanding mean and median}\\
\textbf{Version:} 4323658d.. 2013-10-07
\smallskip\hrule





\section{\href{https://www.khanacademy.org/devadmin/content/items/x65d01dadbdb24ac3}{x65d01dadbdb24ac3}}

\noindent
Over the last $15$ days of August the maximum daily temperatures (in degrees Fahrenheit) in a city were as follows:

$\qquad72,~72,~78,~74,~76,~72,~69,~66,~69,~73,~70,~72,~73,~74,~75$

**Which of *these* temperatures could be used to summarize the observations?**




\paragraph{Ans} [[? input-number 1]] degrees Fahrenheit  72

\paragraph{Hint 1}We could use the middle temperature (the median) to summarize the $15$ temperatures.

\paragraph{Hint 2}The median is the middle temperature when the $15$ temperatures are arranged in order.

$\qquad66,~69,~69,~70,~72,~72,~72,~\green{72},~73,~73,~74,~74,~75,~76,~78$

\paragraph{Hint 3}We could use a temperature of  $72$ degrees Fahrenheit to summarize the observations.



\medskip
\noindent
\textbf{Tags:} {\footnotesize CC.6.SP.A.2, SB.6.1.I.3.CR, Understanding mean and median}\\
\textbf{Version:} 8788ae79.. 2013-10-16
\smallskip\hrule





\section{\href{https://www.khanacademy.org/devadmin/content/items/x6c571dd50d4c11ed}{x6c571dd50d4c11ed}}

\noindent
Babe Buth hit the following number of home runs in each of seasons he played:

$\qquad 54,~59,~35,~41,~46,~25,~47,~60,~54,~46,~49,~46,~41,~34,~22$

**On average he hit  [[? input-number 1]] home runs every season.**  
**The median number of home runs Babe Buth hit is  [[? input-number 2]].**

**Is the average or the median a better choice for the measure of center of this data set?**

\paragraph{Ans} 

The average is a better choice in comparison to the median because the median is too high.

The median is a better choice because the average is low compared to most of the numbers..

\fbox{
\parbox{0.4\textwidth}{ 
 The mean or the median could be used as the measure of center because both are equally good.
 }

}

 

\paragraph{Hint 1}We could work out the average number of home runs per season as follows:

\begin{align*}\text{Mean}&=\dfrac{\text{Sum of home runs in all the seasons}}{\text{Number of seasons played}}\\
\\
&=\dfrac{659}{15}\\
\\
&= 43.9\end{align*}

\paragraph{Hint 2}The median would be the middle number when arranged in order:

$\qquad22,~25,~34,~35,~41,~41,~46,\red{~46,}~46,~47,~49,~54,~54,~59,~60$

$\text{Median}=46$

\paragraph{Hint 3}The mean and the median are almost equal to each other. Therefore, either one may be used as the measure of center.

\paragraph{Hint 4}The mean is $43.9$.  The median is $46$.  The mean or the median could be used as the measure of center because both are equally good.



\medskip
\noindent
\textbf{Tags:} {\footnotesize CC.6.SP.A.3, SB.6.1.I.4.SR, Understanding mean and median}\\
\textbf{Version:} 12709b89.. 2013-10-07
\smallskip\hrule





\section{\href{https://www.khanacademy.org/devadmin/content/items/x6d2380512fbcb2a3}{x6d2380512fbcb2a3}}

\noindent
There are $4$ bookshelves in the school library with $12, ~10, ~16,$ and $18$ books, respectively. 

**How many books would there be on each shelf  if each shelf had the same number of books?**

\paragraph{Ans} [[? input-number 1]] books  14

\paragraph{Hint 1}We can use the mean to balance the number of books on each shelf.

\begin{align*}\text{Mean} &= \dfrac{\text{Sum of books on all shelves}}{\text{Number of shelves}}\\
\\
&= \dfrac{12+10+16+18 }{ 4}\\
\\&=14\end{align*}

\paragraph{Hint 2}Each shelf would have $14$ books.



\medskip
\noindent
\textbf{Tags:} {\footnotesize CC.6.SP.A.2, SB.6.1.I.3.CR, Understanding mean and median}\\
\textbf{Version:} fca3db4d.. 2013-10-07
\smallskip\hrule





\section{\href{https://www.khanacademy.org/devadmin/content/items/x7535a2968731f567}{x7535a2968731f567}}

\noindent
Sam has taken more than one test this year and he is looking at the scores he earned on his tests.  He has never earned the same score more than once. The median of his scores is $88$. 

**Which statement(s) must be true?**

\paragraph{Ans} 

Sam scored $88$ on exactly one test

\fbox{ Sam had as many test scores below $88$ as above

}

 The average of all of Sam's scores is $88$

Sam scored $88$ on all the tests



\paragraph{Hint 1}If there is an odd number of scores, the median is the middle value. One possible list of scores with $88$ as the median is:

$\qquad 32,~88,~89$

\paragraph{Hint 2}We can see that the average is not $88$ in this example.

\paragraph{Hint 3}We can also see that Sam had scores other than $88$.

\paragraph{Hint 4}If there's an even number of scores, the median is the average value of the middle two scores. Another possible list of scores with $88$ as the median is:

$\qquad 47,~86,~90, ~100$

\paragraph{Hint 5}In this example, Sam didn't score exactly $88$ on any test, but the median is still $88$.

\paragraph{Hint 6}In both of these examples, Sam had the same number of scores above and below $88$. It's possible that this isn't true, for example:

$\qquad 73,~88, ~88$

The median is $88$, but since $88$ *appeared more than once*, there aren't any scores above $88$, and there is one score below $88$.

We were told in the problem that he *didn't get any score more than once*, so we can be sure Sam has the same number of scores above and below $88$.

\paragraph{Hint 7}The statement "Sam had as many test scores below $88$ as above" alone must be true.



\medskip
\noindent
\textbf{Tags:} {\footnotesize CC.6.SP.A.2, SB.6.1.I.3.CR, Understanding mean and median}\\
\textbf{Version:} eafba54e.. 2013-10-04
\smallskip\hrule





\section{\href{https://www.khanacademy.org/devadmin/content/items/x76efd161589eec48}{x76efd161589eec48}}

\noindent
John got the following scores on his science tests this year:  

$\qquad80,~85,~85,~90,~85,~85,~95,~85,~80$

While giving the final grade the teacher gives equal weight to each of the tests.

**What score would John's final grade be based on?**

\paragraph{Ans} 

\paragraph{Hint 1}If the teacher gives equal weight to each of the tests, John's final grade would depend on the average of all his test scores.  To find the average of his test scores, we can add the scores and divide the sum by the number of tests.

\paragraph{Hint 2}\begin{align*}\text{Average score}&=\dfrac{\text{Total of all scores}}{\text{Number of tests}}\\
\\
&=\dfrac{770}{9} \\
\\
&= 85.6 \\
\\
\end{align*}

\paragraph{Hint 3}John's final grade would be based on a score of $85.6$.



\medskip
\noindent
\textbf{Tags:} {\footnotesize CC.6.SP.A.2, SB.6.1.I.3.CR, Understanding mean and median}\\
\textbf{Version:} 060c9691.. 2013-10-08
\smallskip\hrule





\section{\href{https://www.khanacademy.org/devadmin/content/items/x79c3f41c2ca154de}{x79c3f41c2ca154de}}

\noindent
The salaries (in millions of dollars) of players on Coston Beltics basketball team are as follows:

$\qquad6.4,~2.5,~1.2,~2.1,~2,~8.7,~12,~5.2,~1.9,~0.5,~12,~1.4,~10$

**The mean salary is $\$$ [[? input-number 1]] millions.**  
**The median salary is $\$$ [[? input-number 2]] millions.**

**Is the mean or the median a better choice for the measure of center of this data set?**

\paragraph{Ans} 

\fbox{
\parbox{0.4\textwidth}{ 
The mean is a better choice in comparison 
to the median because the median is too low.
}
}

 The median is a better choice because the mean is  low compared to most of the salaries.

The mean or the median could be used as the measure of center because both are equally good.



\paragraph{Hint 1}We can arrive at the mean salary by dividing the total of all salaries  by the number of players.

\begin{align*}\text{Mean salary}&=\dfrac{\text{Sum of all salaries}}{\text{Number of salaries}}\\
\\
&=\dfrac{65.9}{13}\\
\\
&= 5.1\end{align*}

\paragraph{Hint 2}The median is the middle salary when the salaries are arranged in order:

$\qquad0.5,~1.2,~1.4,~1.9,~2,~2.1,\red{~2.5,}~5.2,~6.4,~8.7,~10,~12,~12$

$\text{Median salary}=2.5$


\paragraph{Hint 3}Among the $13$ salary figures, there are $7$ salary figures which are lower than the mean salary ($\$5,100,000$), and the other $6$ salary figures are higher than the mean. The median, although it is the middle salary ($\$2,500,000$), is way too low in comparison to all the salaries at the higher end.  Therefore, for this data set mean salary would be better as the measure of center.

\paragraph{Hint 4}The mean salary is $\$5,100,000$.  The median salary is $ \$3,750,000$.  The mean is a better choice in comparison to the median because the median is too low.



\medskip
\noindent
\textbf{Tags:} {\footnotesize CC.6.SP.A.3, SB.6.1.I.4.SR, Understanding mean and median}\\
\textbf{Version:} 9df30b15.. 2013-10-07
\smallskip\hrule





\section{\href{https://www.khanacademy.org/devadmin/content/items/x9100a31b0c60a126}{x9100a31b0c60a126}}

\noindent
In a middle school the following final grades were awarded to a group of $11$ students in mathematics :  

$\qquad\text {C+, B-, B+, C, B+, B-, A-, A+, B, B+, B-}$

**Which grade could be used to describe the overall achievement of the class?**

\paragraph{Ans}  $B$ 

\paragraph{Hint 1}The overall achievement of the class can be described by the mean or the median grade. 

\paragraph{Hint 2}It is not possible to calculate the mean since we do not know the numerical value of the grades.   

\paragraph{Hint 3}The median is the middle grade when all $11$ grades are arranged in order.

$\text{C,  C+, B-, B-, B-,}  \green{\text { B}},\text{ B+, B+, B+, A-, A+}$

\paragraph{Hint 4}The overall class achievement could be described as grade $\text{B}$



\medskip
\noindent
\textbf{Tags:} {\footnotesize CC.6.SP.A.2, SB.6.1.I.3.CR, Understanding mean and median}\\
\textbf{Version:} 01f8304b.. 2013-10-16
\smallskip\hrule





\section{\href{https://www.khanacademy.org/devadmin/content/items/x95c7dae7a87f1163}{x95c7dae7a87f1163}}

\noindent
Justin got the following numbers when he rolled a die $11$ times.

$\qquad1,~6,~2,~4,~3,~2,~1,~6,~2,~1,~6$

**The mean is  [[? input-number 1]].**  
**The median is  [[? input-number 2]].**

**Is the mean or the median a better choice for the measure of center of this data set?**

\paragraph{Ans} 

\fbox{ 
\parbox{0.4\textwidth}{ 
The mean is a better choice in comparison to the median because the median is too low.
}

}

 The median is a better choice because the mean is  high compared to most of the salaries.

The mean or the median could be used as the measure of center because both are equally good.



\paragraph{Hint 1}Let us calculate the mean.

\begin{align*}\text{Mean}&=\dfrac{\text{Sum of all numbers}}{\text{Number of times the die was rolled}}\\
\\
&=\dfrac{34}{11}\\
\\
&= 3.1\end{align*}

\paragraph{Hint 2}The median is the middle number when they are arranged in order:

$\qquad1,~1,~1,~2,~2,\red{~2,}~3,~4,~6,~6,~6$

$\text{Median}=2$

\paragraph{Hint 3}Among the $11$ numbers, there are $7$ numbers which are lower than the mean, and $4$ higher than the mean. The median, although it is the middle number, is too low in comparison to all the numbers at the higher end.  Therefore, for this data set mean would be a better choice as the measure of center.

\paragraph{Hint 4}The mean is $3.1$.  The median is $2$.  The mean is a better choice in comparison to the median because the median is too low.



\medskip
\noindent
\textbf{Tags:} {\footnotesize CC.6.SP.A.3, SB.6.1.I.4.SR, Understanding mean and median}\\
\textbf{Version:} bab39cbe.. 2013-10-07
\smallskip\hrule





\section{\href{https://www.khanacademy.org/devadmin/content/items/xa5f3617d9a0bd7d7}{xa5f3617d9a0bd7d7}}

\noindent
A group of $10$ people have the following number of friends on Facehook

$\qquad131,~168,~268,~130,~144,~237,~159,~146,~286,~157$

**Which of *these* numbers could best summarize the data?**


\paragraph{Ans} 

\paragraph{Hint 1}We could summarize the data by using the average or the median. 

\paragraph{Hint 2}\begin{align*}\text{Average number of friends} &= \dfrac{\text{Total number of friends }}{\text{Number of people}}\\ 
\\
&= \dfrac{1826}{10}\\
\\
&=182.6\end{align*}

But $182.6$ is not part of the original set of numbers.

\paragraph{Hint 3}In this case the median would be the average of the two middle numbers, when the numbers are arranged in order.

\paragraph{Hint 4}$\qquad130,~131,~144,~146,\green{~157,~159},~168,~237,~268,~286$

\begin{align*}\text{Median}&=\dfrac{157+159}{2}\\
\\
&=158\end{align*}

\paragraph{Hint 5}We could use $158$ to summarize the given data .



\medskip
\noindent
\textbf{Tags:} {\footnotesize CC.6.SP.A.2, SB.6.1.I.3.CR, Understanding mean and median}\\
\textbf{Version:} 3a8a8ebe.. 2013-10-16
\smallskip\hrule





\section{\href{https://www.khanacademy.org/devadmin/content/items/xa93386ad45c17b68}{xa93386ad45c17b68}}

\noindent
A group of $10$ people have the following number of friends on Facehook

$\qquad131,~168,~268,~130,~144,~237,~159,~146,~286,~157$

**On average how many friends does each person have on facehook?**


\paragraph{Ans} 

\paragraph{Hint 1}We can calculate the average by dividing the total number of friends by the number of people.

\paragraph{Hint 2}\begin{align*}\text{Average number of friends} &= \dfrac{\text{Total number of friends }}{\text{Number of people}}\\ 
\\
&= \dfrac{1826}{10}\\
\\
&=182.6\end{align*}


\paragraph{Hint 3}This group of people on average has $182.6$ friends on facehook.



\medskip
\noindent
\textbf{Tags:} {\footnotesize CC.6.SP.A.2, SB.6.1.I.3.CR, Understanding mean and median}\\
\textbf{Version:} 036ad61d.. 2013-10-16
\smallskip\hrule





\section{\href{https://www.khanacademy.org/devadmin/content/items/xb9001c5c9b794f88}{xb9001c5c9b794f88}}

\noindent
In the Men's singles of a tennis tournament, the number of sets played in each of the matches were as follows:

\begin{align*}\qquad\text{Pre-Quarters}&: 5,~4,~3,~4,~3,~5,~4,~4\\
\text{Quarters}&: 4,~3,~5,~3\\ \text{Semis}&:5,~3\\ \text{Final}&:4\end{align*}

**Typically, how many sets did a match last?**

Give your answer as a whole number.



\paragraph{Ans} [[? input-number 1]] sets  4

\paragraph{Hint 1}Let us begin by arranging the number of sets in order.  The median would be the middle number.

$\qquad3,~3,~3,~3,~3,~ 4,~4,~\red{4},~4,~4,~4,~ 5,~5,~5,~5$

\paragraph{Hint 2}Since there are $5$ matches with $3$ sets each, $6$ matches with $4$ sets each, and $4$ matches with $5$ sets each, we can estimate that the average number of sets per match is very also close to $4$.

\paragraph{Hint 3}We can say that typically, a match lasted $4$ sets.



\medskip
\noindent
\textbf{Tags:} {\footnotesize CC.6.SP.A.2, SB.6.1.I.3.CR, Understanding mean and median}\\
\textbf{Version:} 1df62f58.. 2013-10-16
\smallskip\hrule





\section{\href{https://www.khanacademy.org/devadmin/content/items/xbbb6d2164d7f5df8}{xbbb6d2164d7f5df8}}

\noindent
The average annual rainfall in a certain region is $30$ inches.  Last year it rained only $25$ inches.  A farmer would like to know about the rainfall this year since it has not rained much so far.

**Choose the correct statement.**

\paragraph{Ans} 

This year it would rain exactly $30$ inches.

It would rain more than $30$ inches this year because last year it rained less. 

It would rain less than $30$ inches this year just like it did last year.

\fbox{ It is impossible to predict anything for sure.

}

 

\paragraph{Hint 1}The average annual rainfall figure is based on data from many past years.  The variation from year to year could be relatively small (e.g. $32,~30,~28$) or could be relatively large (e.g. $20,~25,~30,~35,~40$).

\paragraph{Hint 2}Since the rainfall varies from year to year, it is not possible to predict anything regarding this year's rainfall just because the average is $30$ inches, or because the rainfall last year was only $25$ inches. The rainfall this year (or in any particular year) could be less than, equal to, or more than $30$ inches.

\paragraph{Hint 3}"It is impossible to predict anything for sure." is the correct statement.



\medskip
\noindent
\textbf{Tags:} {\footnotesize CC.6.SP.A.2, SB.6.1.I.3.CR, Understanding mean and median}\\
\textbf{Version:} 32db0463.. 2013-10-07
\smallskip\hrule





\section{\href{https://www.khanacademy.org/devadmin/content/items/xbd23751064c9d7e7}{xbd23751064c9d7e7}}

\noindent
In a leap year, the days in a month vary from January to December as shown in the table:  

**How many days does the average month have in a leap year?**

Month | Days | 
- | - | -
January | $31$ | 
February | $29$ | 
March | $31$ | 
April | $30$ | 
May | $31$ | 
June | $30$ | 
July | $31$ | 
August | $31$ | 
September | $30$ | 
October | $31$ | 
November | $30$ | 
December | $31$ | 


\paragraph{Ans} [[? input-number 1]] days  30.5

\paragraph{Hint 1}To have the same number of days every month, we would have to divide the total number of days in a leap year among the $12$ months equally.  

\paragraph{Hint 2}\begin{align*}\text{The average number of days} &= \dfrac{\text{Total number of days in a leap year}}{\text{Number of months}}\\ 
\\
&= \dfrac{366}{12}\\
\\
&=30.5\end{align*}

\paragraph{Hint 3}On average there are $30.5$ days in a month in a leap year.



\medskip
\noindent
\textbf{Tags:} {\footnotesize CC.6.SP.A.2, SB.6.1.I.3.CR, Understanding mean and median}\\
\textbf{Version:} 74e3c524.. 2013-10-09
\smallskip\hrule





\section{\href{https://www.khanacademy.org/devadmin/content/items/xcb2fe394520941c9}{xcb2fe394520941c9}}

\noindent
A group of $6$ children have with them $12, ~24, ~23, ~17, ~18$ and $20$ marbles, respectively.  

**How many marbles would each child have if the marbles were to be shared equally?**


\paragraph{Ans} [[? input-number 1]] marbles  19

\paragraph{Hint 1}We can use the mean to find the number of marbles with each child, if the marbles were to be shared equally.

\begin{align*}\text{Mean} &= \dfrac{\text{Sum of marbles with all children }}{\text{Number of children}}\\
\\
 &= \dfrac{12 +24+23+17+18+20 }{6}\\
&=19\end{align*}

\paragraph{Hint 2}Each child would have $19$ marbles.



\medskip
\noindent
\textbf{Tags:} {\footnotesize CC.6.SP.A.2, SB.6.1.I.3.CR, Understanding mean and median}\\
\textbf{Version:} 44b4b921.. 2013-10-07
\smallskip\hrule





\section{\href{https://www.khanacademy.org/devadmin/content/items/xcf651f72ebbca86e}{xcf651f72ebbca86e}}

\noindent
Below are starting salaries (in thousands of dollars) of some of the graduates from Stanward University this year.

$\qquad40,~42,~44,~50,~50,~51,~51,~51,~52,~52,~52,~52,~52,~54,~54$

**The mean starting salary is $\$$ [[? input-number 1]] thousand.**  
**The median starting salary is $\$$ [[? input-number 2]] thousand.**

**Is the mean or the median a better choice for the measure of center of this data set?**

\paragraph{Ans} 

The mean is a better choice because the salaries are almost equally spread around the mean.

\fbox{
\parbox{0.4\textwidth}{ 
 The median is a better choice because the mean is  low compared to most of the salaries.
 }

}

 The mean or the median could be used as the measure of center because both are equally good.




\paragraph{Hint 1}We would get the mean salary by dividing the sum of all salaries by the number of salaries.

\begin{align*}\text{Mean salary}&=\dfrac{\text{Sum of all salaries}}{\text{Number of salaries}}\\
\\
&=\dfrac{747}{15}\\
\\
&= 49.8\end{align*}

\paragraph{Hint 2}The median is the middle salary when the salaries are arranged in order:

$\qquad40,~42,~44,~50,~50,~51,~51,\red{~51},~52,~52,~52,~52,~52,~54,~54$

\paragraph{Hint 3}Among the $15$ starting salaries, there are only $3$ salaries which are lower than the mean salary ($\$49,800$), and the other $12$ salaries are higher than the mean. Therefore, for this data set the median salary ($\$51,000$) would be a better choice for the measure of center compared to the mean salary.

\paragraph{Hint 4}The mean salary is $\$49,800$.  The median salary is $ \$51,000$. The median is a better choice because the mean is  low compared to most of the salaries.



\medskip
\noindent
\textbf{Tags:} {\footnotesize CC.6.SP.A.3, SB.6.1.I.4.SR, Understanding mean and median}\\
\textbf{Version:} 84bea1fc.. 2013-10-07
\smallskip\hrule





\section{\href{https://www.khanacademy.org/devadmin/content/items/xf044e619fc062683}{xf044e619fc062683}}

\noindent
A group of $5$ children have with them $~2, ~4, ~3, ~4,$ and $2$ cookies, respectively.  

**How many cookies would each child have if the children decided to share all the cookies equally?**


\paragraph{Ans} [[? input-number 1]] cookies  3

\paragraph{Hint 1}We can use the average to make the number of cookies with each child the same.

\begin{align*}\text{Average} &= \dfrac{\text{Sum of cookies with all children}}{\text{Number of children}}\\
\\
&= \dfrac{2+4+3+4+2 }{5}\\
&=3\end{align*}

\paragraph{Hint 2}Each child would have $3$ cookies.



\medskip
\noindent
\textbf{Tags:} {\footnotesize CC.6.SP.A.2, SB.6.1.I.3.CR, Understanding mean and median}\\
\textbf{Version:} 3697060d.. 2013-10-07
\smallskip\hrule





\section{\href{https://www.khanacademy.org/devadmin/content/items/xfa7d55dde2bd8d62}{xfa7d55dde2bd8d62}}

\noindent
At $5$ different gas stations the gas prices (in $\$$ per gallon) were:  

$\qquad4.05, ~4.06, ~4.09, ~4.10, ~4.25$

**What was the average gas price for these $5$ gas stations?**

\paragraph{Ans} $\$$[[? input-number 1]] per gallon  4.1

\paragraph{Hint 1}We could get the average price by dividing the total of prices at all the gas stations by the number of gas stations.

\paragraph{Hint 2}\begin{align*}\text{Average} &= \dfrac{\text{Total of all the prices}}{\text{Number of gas stations}}\\
\\
&= \dfrac{4.05+4.06+4.09+4.10+4.25}{5}\\
&=4.10\end{align*}

\paragraph{Hint 3}The average gas price was $\$4.10$ per gallon.



\medskip
\noindent
\textbf{Tags:} {\footnotesize CC.6.SP.A.2, SB.6.1.I.3.CR, Understanding mean and median}\\
\textbf{Version:} 9eeb6177.. 2013-10-07
\smallskip\hrule





\section{\href{https://www.khanacademy.org/devadmin/content/items/xfcb23dc96da2484b}{xfcb23dc96da2484b}}

\noindent
There are $5$ identical cylindrical jars filled with water up to levels of $6, ~4, ~9, ~5,$ and $6$ cm, respectively.  

**What would the level of water be in each jar if the water were to be equally distributed?**

\paragraph{Ans} [[? input-number 1]] cm  6

\paragraph{Hint 1}We can use the mean to make all the levels equal to each other. 

\begin{align*}\text{Mean} &= \dfrac{\text{Sum of levels in all jars}}{\text{Number of jars}}\\
\\
&= \dfrac{ 6+4+9+5+6 }{5}\\
&=6\end{align*}

\paragraph{Hint 2}The level would be $6$ cm.



\medskip
\noindent
\textbf{Tags:} {\footnotesize CC.6.SP.A.2, SB.6.1.I.3.CR, Understanding mean and median}\\
\textbf{Version:} ee0787df.. 2013-10-07
\smallskip\hrule





\section{\href{https://www.khanacademy.org/devadmin/content/items/xfdc6c4d33b069fcd}{xfdc6c4d33b069fcd}}

\noindent
During the regular football season $2012$, the points scored by the $39$ers in their games were as follows:

$\qquad30,~27,~13,~34,~45,~3,~13,~24,~24,~32,~31,~13,~27,~41,~13,~27$

**Which of *these* numbers would best summarize the season's performance?**

\paragraph{Ans} 

\paragraph{Hint 1}We could use the average or the median to summarize the season's performance

\paragraph{Hint 2}\begin{align*}\text{Average points per game} &= \dfrac{\text{Total points scored in all the games}}{\text{Number of games}}\\ 
\\
&= \dfrac{397}{16}\\
\\
&=24.8\end{align*}

However, the $39$ers did not score $24.8$ points in any of their games.

\paragraph{Hint 3}Let's calculate the median.  The number of games is $16$.  Therefore, the median would be the average of the two middle numbers  when the points scored in all the games are arranged in order.

\paragraph{Hint 4}$\qquad3,~13,~13,~13,~13,~24,~24,\green{~27,~27},~27,~30,~31,~32,~34,~41,~45$

The average of $27,$ and $27$, is $27$.

\paragraph{Hint 5}From the given numbers, $27$ could best summarize the season's performance.



\medskip
\noindent
\textbf{Tags:} {\footnotesize CC.6.SP.A.2, SB.6.1.I.3.CR, Understanding mean and median}\\
\textbf{Version:} f624e369.. 2013-10-16
\smallskip\hrule



%%  Create a directory called 'figures' in latex dir and run the following command 
%  wget -N \


\end{document}


