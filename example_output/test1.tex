\documentclass[10pt]{article}
\title{Khan exercises}
 \usepackage{amsmath,hyperref}
 \usepackage[usenames,dvipsnames]{color}
 
 \newcommand{\blue}[1]{{\color{Blue}#1}} 
 \newcommand{\purple}[1]{{\color{Purple}#1}} 
 \newcommand{\red}[1]{{\color{Red}#1}} 
 \newcommand{\green}[1]{{\color{Green}#1}} 
 \newcommand{\gray}[1]{{\color{Gray}#1}} 


\begin{document}
\maketitle



\section{\href{https://www.khanacademy.org/devadmin/content/items/x4eabf393}{x4eabf393}}

Jimmy will be selling hot dogs at the football game. He bought hot dogs, buns and condiments for $\$8$ before the game and now he wants to calculate the profit he will make. The graph below shows how Sean's profit $P\:$ depends on the number of hot dogs he sells at the game.

![](https://ka-perseus-graphie.s3.amazonaws.com/2d882eb9739295abe5dc10510ecf8f2564fd633e.png)

**Write the equation which describes Jimmy's profit $P\:$ as a function of the number $n$ of hot dogs sold.**

The equation which describes Jimmy's profit $P$ as a function of the number of hot dogs sold $n$ is   
[[? expression 1]]

\paragraph{Hint 1}The graph shows how the profit $\blue{P}\:$ varies as a function of the number of hot dogs sold $\red{n}$. Let's see how to use the graph of the function to find the equation of the function.

\paragraph{Hint 2}Looking at the graph, we see the initial value of the profit function is $\green{-8}$ dollars. If Jimmy doesn't sell any hot dogs ($\red{n}=0$), his profit will be $\green{-8}$ dollars. A negative profit is a *loss*. 
Indeed, if Jimmy doesn't sell any hot dogs he will lose the money he invested to buy the ingredients.

\paragraph{Hint 3}Next, observe the slope of the graph is equal to $\purple{1}$. For each hot dog sold, the profit increases by $\$\purple{1}$. This means Jimmy is selling the hot dogs for $\$\purple{1}$ each.

\paragraph{Hint 4}Combining these facts about Jimmy's hot dog operation, we can now write his profit $\blue{P}$ as a function of the number of sales $\red{n}$ as follows:

\begin{align*}
\quad \blue{P} 
&=  \green{-8} + \purple{1}\red{n}.
\end{align*}


\paragraph{Hint 5}Note that the profit $\blue{P}\:$ is described by a *linear equation* $\blue{P}=\purple{m}\cdot\red{n}  + \green{b}$, where $\green{b}=\green{-8}$ represents Jimmy's initial investment and $\purple{m}=\purple{1}$ represents sale price of each hot dog.

The graph of the function $\blue{P}=\green{-8} + \purple{1}\red{n}$ is a *line* which passes through the point $(\red{0},\blue{-8})$ and has slope equal to $\purple{1}$. We can understand the function which describes Jimmy's profit either through its equation or through its graph.

\paragraph{Hint 6}The equation which describes Jimmy's profit $P$ as a function of the number of hot dogs sold $n$ is  $P = -8+1\cdot n$.





\section{x4d8feb39}\url{https://www.khanacademy.org/devadmin/content/items/x4d8feb39}

Sean works in sales. His monthly salary $S$ depends on his sales performance. The graph below shows his salary as a function of the number of sales he makes during the month.

![](https://ka-perseus-graphie.s3.amazonaws.com/88f55c935fe40d38b0894522ee4c9ee3e18982d9.png)

**Find the equation which describes Sean's monthly salary $S$ as a function of the number of sales $n$.**

The equation which describes Sean's salary $S$ as a function of the number of sales $n$ is  
[[? expression 1]]

\paragraph{Hint 1}We�re shown the graph of the function which describes how Sean's monthly salary $\blue{S}$ depends on the number of sales $\red{n}$. Let's use the information from the graph to figure out the equation which represents $\blue{S}$ as a function of $\red{n}$.

\paragraph{Hint 2}Looking at the graph we see that the initial value of the function $\blue{S}$ is $\$\green{1000}$. Even if Sean makes $\red{n}=0$ sales, he will still receive a monthly salary of $\$\green{1000}$. 

\paragraph{Hint 3}Next, let's look at the slope of the graph. If Sean makes $\red{x}=\red{2}$ sales, his salary will be $\blue{S}=\$\blue{1200}$. Thus, a change of $\red{2}$ in the number of sales $\red{n}$ produces a change of $\$\blue{200}$ in the salary $\blue{S}$. The rate of change of the salary function is:

$\quad 
\purple{m} = \dfrac{ \textrm{change in } \blue{S} }{ \textrm{change in } \red{n} } = 
\dfrac{ \blue{1200} - \blue{1000} }{ \red{2} - \red{0} } 
= \dfrac{200}{2} = \purple{100}$.

Sean's salary increases by $\$\purple{100}$ for each sale he makes. 

\paragraph{Hint 4}Combining all the information we have about Sean's monthly salary, we can now write his salary $\blue{S}$ as a function of the number of sales $\red{n}$ as follows:

\begin{align*}
\quad \blue{S} 
&=  \green{1000} + \purple{100}\red{n}.
\end{align*}

\paragraph{Hint 5}Note that the salary $\blue{S}$ is described by a *linear equation* $\blue{S}=\purple{m}\cdot\red{n}  + \green{b}$, where $\green{b}=\green{1000}$ represents the initial value of the function and $\purple{m}=\purple{100}$ represents the rate of change of the function.

The graph of the function $\blue{S}=\green{1000} + \purple{100}\red{n}$ is a *line* which passes through the point $(\red{0},\blue{1000})$ and has slope equal to $\purple{100}$. We can understand the function which describes Sean's salary either through its equation or through its graph.

\paragraph{Hint 6}The equation which describes Sean's salary $S$ as a function of the number of sales $n$ is  $S = 1000+100n$.





\section{x464f53e1}\url{https://www.khanacademy.org/devadmin/content/items/x464f53e1}

Jessica works in sales. Her monthly salary is calculated as a base amount of $\$2000$ plus a commission of $\$100$ for each sale she closes. Assume $n$ represents the number of sales Jessica closes and $S$ represents her monthly salary.

**Find the equation which represents Jessica's monthly salary $S$ as a function of the number of sales $n$.**

The equation which describes $S$ as a function of $n$ is   
[[? expression 1]]

\paragraph{Hint 1}We are told Jessica's salary $\blue{S}$ increases by $\$\purple{100}$ for each sale she makes. Also, we know that she receives a base amount of $\$\green{2000}$.  Let's see how we can use the information provided to figure out the function which describes Jessica's monthly salary.

\paragraph{Hint 2}The initial value of the salary function is $\$\green{2000}$. This is the base amount Jessica earns even when she makes $\red{n}=0$ sales.

The rate of change of the salary function is $\$\purple{100}$ per sale because this is how much she makes per sale. If $\red{n}$ increases by one, her salary $\blue{S}$ will increase by $\$\purple{100}$. 

We can combine these two facts to find Jessica's monthly salary $\blue{S}$ as a function of the number of sales $\red{n}$:

 \begin{align*}
\quad \blue{S} &= \purple{100}\red{n}  + \green{2000}.
\end{align*}

\paragraph{Hint 3}Note that the salary $\blue{S}$ is described by a *linear equation* $\blue{S}=\purple{m}\cdot\red{n}  + \green{b}$, where $\green{b}=\green{2000}$ represents the initial value of the function and $\purple{m}=\purple{100}$ represents the rate of change of the function.

\paragraph{Hint 4}The equation which describes $S$ as a function of $n$ is $S= 2000+100n$.





\section{xd3aa970e}\url{https://www.khanacademy.org/devadmin/content/items/xd3aa970e}

Ron borrowed $\$200$ from his friend and promised to return the money by paying back $\$20$ each week. Assume $x$ represents the time in weeks, and $y$ represents the amount of money left to pay.

**Find the equation which describes the loan remaining $L$ as a function of time $t$.**

Drag the two points to move the line into the correct position. 

\paragraph{Hint 1}We�re looking for the graph of the function which describes the amount of money $\blue{y}$ remaining  for Ron to pay back to his friend as a function of the time $\red{x}$ measured in weeks.

\paragraph{Hint 2}The initial value of Ron's debt is $\$\green{200}$. This is the amount Ron has to pay back. Ron's debt $\blue{y}$ decreases by $\$\purple{20}$ each week.
Ron's debt is described by the following linear equation:

 \begin{align*}
\quad \blue{y} 
 &= \purple{m}\cdot\red{x}  + \green{b} \\[1mm]
 &= \purple{-20}\cdot\red{x}  + \green{200}.
\end{align*}

The initial value of the loan is $\green{b=200}$. This is the amount Ron owes to his friend when $\red{x=0}$.
The rate of change of the function is $\purple{m=-20}$ because each week Ron pays back $\$\purple{20}$ to his friend.

\paragraph{Hint 3}The graph of the function $\blue{y}=\purple{m}\cdot\red{x}+\green{b}$ is a line with $y$-intercept equal to $\green{b}$ and slope equal to $\purple{m}$.

Therefore, the graph of Ron's debt $\blue{y} = \green{200}  \purple{-20}\red{x}$ is a line which passes through the point $(\red{0},\blue{200})$ and has slope equal to $\purple{-20}$:   
![](https://ka-perseus-graphie.s3.amazonaws.com/bbe5bcc71845e5f1f2200362541175a6bb963515.png)





\section{xe3e0bf01}\url{https://www.khanacademy.org/devadmin/content/items/xe3e0bf01}

The cost of your annual visit to the dentist is calculated as a base price of $\$50$ dollars for the checkup and cleaning plus an additional $\$100$ per cavity the dentist has to fix.

**What is the equation which describes the cost $C$ of the visit if the dentist finds $n$ cavities?**

The cost of the visit $C$ as a function of the number of cavities $n$ is described by
[[? expression 1]]

\paragraph{Hint 1}(draft)

Don't worry, you are not at the dentist�this is just a math question!

The cost of the visit, in dollars, is described by the math expression $\green{50} + \red{100}\purple{n}$, where $\$50$ is the base price for the visit and $\$100$ is the price for repairing one cavity.

We have to evaluate this expression in the case of $\purple{n=2}$ cavities.

\paragraph{Hint 2}Plugging in the value $\purple{n=2}$ into the formula $\green{50} + \red{100}\purple{n}$ we obtain the following expression:  

\begin{align*}
\green{50} + \red{100}\purple{n}
 &= \green{50} + \red{100}\green{\cdot}\purple{2}\\
 &= 50 \green{+} 200 \\
 &= 250
\end{align*}

Note the order in which we performed the operations when evaluating the expression. We computed the product before carrying out the addition.

\paragraph{Hint 3}The cost of the visit to the dentist will be $\$250$.





\section{x66af7067}\url{https://www.khanacademy.org/devadmin/content/items/x66af7067}

Covi is driving from Montreal to New York City. The graph below shows his position $x$, measured in kilometers from the Canada-US border, as a function of time $t$, measured in hours.

![](https://ka-perseus-graphie.s3.amazonaws.com/e8fa2c2d06cffc6ef871bbc124814ca2d111357d.png)

**What is the equation that describes $x$ as a function of $t$?**

The equation which describes $x$ as a function of $t$ is 
[[? expression 1]]

\paragraph{Hint 1}We are shown the graph of Covi's position function, and we want to find the equation that corresponds to this graph. The graph we see is a line. Observe that the slope of this line is equal to $\purple{100}$ and its $y$-intercept is $\green{-200}$.

\paragraph{Hint 2}If the graph of a function looks like line, then the function is described by a linear equation $\blue{y} = \purple{m}\red{x} + \green{b}$, where $\purple{m}$ is the *rate of change* of the function and $\green{b}$ is the *initial value* of the function.  To find the equation of the function, we must figure out the values of $\purple{m}$ and $\green{b}$.

\paragraph{Hint 3}The graph of this line passes through the point $(0,\green{-200})$. We say the $y$-intercept of the graph is equal to $\green{-200}$ because this is where the graph crosses the $y$-axis.

Let�s now think in terms of the equation $\blue{x} = \purple{m}\red{t} + \green{b}$. When the input is $\red{t}=\red{0}$ the output is $\blue{x}=\blue{-200}$:  

\begin{align*}
\qquad \blue{y} & = \purple{m}\red{x} + \green{b} \\
4 & = m(0) +\green{b} \\
4 & = \green{b}
\end{align*}    

So $\green{b}=\green{4}$. The initial value of the function corresponds to the $y$-intercept of its graph.

\paragraph{Hint 4}If we can find the value of the rate of change $\purple{m}$ of the function we will be done. Observe that the graph passes through the point $(\red{1},\blue{6})$. When $\red{x}=\blue{1}$, $\blue{y}=\blue{6}$. Let�s plug these values into the equation:

\begin{align*}
\qquad \blue{y} & = \purple{m}\red{x} + \green{b} \\
\blue{6} & = \purple{m}(\red{1}) + \green{4} \\
6 & = \purple{m}  + 4 \\
2 & =\purple{m}       \\
\end{align*}  

So $\purple{m}=\purple{2}$.

Now we can write the equation of the function whose graph we see in the figure:

$\blue{y} = \purple{2}\red{x} + \green{4}$

The rate of change $\purple{m}=\purple{2}$ corresponds to the slope of the graph, and the initial value $\green{b}=\green{4}$ corresponds to the the $y$-intercept of the graph.

\paragraph{Hint 5}The equation which describes $y$ as a function of $x$ is $y = 2x + 4$.





\section{x288363ab}\url{https://www.khanacademy.org/devadmin/content/items/x288363ab}

Jane runs a company which manufactures bicycles. The operating costs of Jane�s company are of two types. Each month, the company must cover the fixed costs of $\$3000$ for rent and utilities. The production costs are $\$100$ per bicycle.

**What is the equation which describes the costs $C$ as a function of the number of bicycles produced $x$?**

The equation that describes $C$ as a function of $x$ is  
 [[? expression 1]]

\paragraph{Hint 1}We�re trying to find the equation which corresponds to the operating costs $\blue{C}$ of Jane�s company as a function of the number of bicycles produced $\red{x}$.
We know the fixed costs are $\$\green{3000}$ per month. Additionally, there is a production cost of $\$\purple{100}$ per bicycle.

\paragraph{Hint 2}The *fixed costs* of a company is the part of the costs that are present even when Jane�s company does not produce any items $\red{x}=0$. We know that each month Jane has to pay $\green{3000}$ dollars for rent and utilities and that this number does not depend on the number of bicycles produced.

\paragraph{Hint 3}Now let's look at the variable costs. The *variable* costs of a company depend on the number of bicycles produced.

If Jane produces $\red{x}$ bicycles this month, her variable costs will be $\purple{100}\red{x}$ dollars, since each bicycle costs $\$\purple{100}$ to produce. 

\paragraph{Hint 4}The total of the costs of Jane�s company is the sum of the fixed costs and the variable costs:

$\quad \blue{C}= \purple{100}\red{x} + \green{3000}$ dollars.

\paragraph{Hint 5}Note that the total costs of the company are described by a linear equation of the form $\blue{C}=\purple{m}\red{x}+\green{b}$, where the number $\purple{m}$ corresponds to the *unit cost of production* the function as $\red{x}$ increases and $\green{b}$ is the *initial value* of the function.

\paragraph{Hint 6}The equation that describes $C$ as a function of $x$  is $C = 100x + 3000$.





\section{x0b401d79}\url{https://www.khanacademy.org/devadmin/content/items/x0b401d79}

David is visiting a new city. He wants to rent a bicycle for a couple of hours to be able to better explore the city. The price he has to pay is $\$4$ per hour plus a base charge of $\$10$.  

**Write the function which represents the price $P\:$ he will have to pay for $h$ hours?**    

$P=$ [[? expression 1]] dollars

\paragraph{Hint 1}The price function corresponds the equation of a line  $\blue{P}=\purple{m}\red{h}+\green{b}$, where the $\blue{m}$ is the *slope* of the function and $\red{b}$ is the *initial value*.

\paragraph{Hint 2}The number $\green{b}$ corresponds to the fixed price of the bike rental, which is $\$\green{10}$.

\paragraph{Hint 3}The number $\purple{m}$ corresponds to the cost of the bike rental per hour. If David rents the bicycle for $\red{h}$ hours, then the price will be $4\red{h}$ dollars plus the base charge.

\paragraph{Hint 4}The price function is $P= 4h + 10$ dollars.





\section{x3d0ab2da}\url{https://www.khanacademy.org/devadmin/content/items/x3d0ab2da}

The graph below shows how the quantity $y$ is related to the quantity $x$.

![](https://ka-perseus-graphie.s3.amazonaws.com/c55256c02b190a7aba6f2a02fb7d6bf49a8652da.png)

**What is the equation that describes $y$ as a function of $x$?**

The equation which describes $y$ as a function of $x$ is 
[[? expression 1]]

\paragraph{Hint 1}We are shown the graph of a function, and we want to find the equation of the function which corresponds to this graph. The graph we see is a line. Observe that the slope of this line is equal to $\purple{2}$ and its $y$-intercept is $\green{-4}$.

\paragraph{Hint 2}If the graph of a function looks like line, then the function is described by a linear equation $\blue{y} = \purple{m}\red{x} + \green{b}$, where $\purple{m}$ is the *rate of change* of the function and $\green{b}$ is the *initial value* of the function.  To find the equation of the function, we must figure out the values of $\purple{m}$ and $\green{b}$.

\paragraph{Hint 3}The graph of this line passes through the point $(0,\green{4})$. We say the $y$-intercept of the graph is equal to $\green{4}$ because this is where the graph crosses the $y$-axis.

Let�s now think in terms of the equation $\blue{y} = \purple{m}\red{x} + \green{b}$. When the input is $\red{x}=\red{0}$ the output is $\blue{y}=\blue{4}$:  

\begin{align*}
\qquad \blue{y} & = \purple{m}\red{x} + \green{b} \\
4 & = m(0) +\green{b} \\
4 & = \green{b}
\end{align*}    

So $\green{b}=\green{4}$. The initial value of the function corresponds to the $y$-intercept of its graph.

\paragraph{Hint 4}If we can find the value of the rate of change $\purple{m}$ of the function we will be done. Observe that the graph passes through the point $(\red{1},\blue{6})$. When $\red{x}=\blue{1}$, $\blue{y}=\blue{6}$. Let�s plug these values into the equation:

\begin{align*}
\qquad \blue{y} & = \purple{m}\red{x} + \green{b} \\
\blue{6} & = \purple{m}(\red{1}) + \green{4} \\
6 & = \purple{m}  + 4 \\
2 & =\purple{m}       \\
\end{align*}  

So $\purple{m}=\purple{2}$.

Now we can write the equation of the function whose graph we see in the figure:

$\blue{y} = \purple{2}\red{x} + \green{4}$

The rate of change $\purple{m}=\purple{2}$ corresponds to the slope of the graph, and the initial value $\green{b}=\green{4}$ corresponds to the the $y$-intercept of the graph.

\paragraph{Hint 5}The equation which describes $y$ as a function of $x$ is $y = 2x + 4$.





\section{xfb15c7a4}\url{https://www.khanacademy.org/devadmin/content/items/xfb15c7a4}

Norm works in marketing. The table below shows the monthly price Norm pays for sending out emails to potential clients.

| $n$ of emails | price $P$|
| :-: | :-: |
| $0$    | $\$10$     |
| $100$    | $\$11$     |
| $200$    | $\$12$     |
| $300$    | $\$13$     |
| $500$    | $\$15$    |
| $1000$    | $\$20$   |
| $3000$    | $\$30$    |


The equation that describes $P$ as a function of $n$ is [[? expression 1]]

\paragraph{Hint 1}We want to find the equation which corresponds to the monthly price $\blue{P}$ as a function of the number of emails sent $\red{n}$. We can look at the values if the table to figure out the equation of the function.

\paragraph{Hint 2}The more emails Observe that the *change* in $\blue{P}$ is proportional to the *change* in $\red{n}$. Going from $\red{n}=100$ emails to $\red{n}=200$, a change of $100$ in $\red{n}$, corresponds to a price increase of $\$1$ dollar. If Paul wants to send $\red{n}=300$, then the monthly price will increase by an additional $\$1$ dollar. 

The rate of change of the price $\blue{P}$ as the number of emails $\red{n}$ increases is equal to:

$\quad \purple{m} = \frac{\$1}{100} = \frac{1}{100}$ dollars per email sent.

Let�s make sure that the *price per email sent* remains the same when Norm sends more emails. Who knows, maybe there is a discount for large volumes of emails. When $\red{n}$ changes from $\red{n}=1000$ to $\red{n}=2000$, the price increases from $\$20$ to $\$30$. 
This change in the price is is consistent with the rate of $\purple{m}=\purple{\frac{1}{100}}$ dollars per email:

$\purple{m}\cdot 1000 =  \purple{1}{100} \cdot 1000 = 10 = 30 - 20 = $ change in price.


The price $\blue{P}\:$ increases at a rate of $\purple{m}=\purple{\frac{1}{100}}$ dollars per email sent.

\paragraph{Hint 3}So can we figure out what the price function is now? 
Since the change in the price $\blue{P}\:$ is proportional to the change in the input $\red{n}$ we know the price is described by a linear equation $\blue{P}=\purple{m}\red{n}+\green{b}$.
The constant $\blue{m}$ corresponds to the *rate of change* of the price and $\green{b}$ is the *initial value* of the function.

We already figured out that $\purple{m}=\purple{\frac{1}{100}}$ dollars per email, so what remains to do is find $\green{b}$, the initial value of the function.

\paragraph{Hint 4}The number $\green{b}$ corresponds to price Norm has to pay even when he sends $\red{n}=0$ emails. When $\red{x}=\red{0}$ the price is $\blue{P} =10 = \purple{m}(0) +\green{b}$,  
so $\green{b}=\green{3000}$. 

\paragraph{Hint 5}The price (in dollars) of sending $\red{n}$ emails is therefore described the following linear equation:

$\quad \blue{P}= \purple{\frac{1}{100}}\red{n} + \green{10}$.



\end{document}