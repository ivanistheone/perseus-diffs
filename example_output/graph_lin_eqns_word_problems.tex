\documentclass[twocolumn,10pt]{article}
\title{Graphing linear equations and word problems}
 \usepackage{amsmath,hyperref,cancel}
 \usepackage[margin=1.5cm]{geometry}
 \usepackage[usenames,dvipsnames]{color}
\setlength{\columnsep}{20pt} 
 \newcommand{\blue}[1]{{\color{Blue}#1}} 
 \newcommand{\purple}[1]{{\color{Purple}#1}} 
 \newcommand{\red}[1]{{\color{Red}#1}} 
 \newcommand{\green}[1]{{\color{Green}#1}} 
 \newcommand{\gray}[1]{{\color{Gray}#1}} 
 \newcommand{\pink}[1]{{\color{Magenta}#1}} 


\begin{document}
\maketitle

\noindent
\href{https://www.khanacademy.org/devadmin/content/items#search=tag%3Aag5zfmtoYW4tYWNhZGVteXIxCxIRQXNzZXNzbWVudEl0ZW1UYWciATAMCxIRQXNzZXNzbWVudEl0ZW1UYWcY0cEZDA%2Ctag%3Aag5zfmtoYW4tYWNhZGVteXIxCxIRQXNzZXNzbWVudEl0ZW1UYWciATAMCxIRQXNzZXNzbWVudEl0ZW1UYWcYuckZDA%2Ctag%3Aag5zfmtoYW4tYWNhZGVteXIxCxIRQXNzZXNzbWVudEl0ZW1UYWciATAMCxIRQXNzZXNzbWVudEl0ZW1UYWcYodEZDA%2Ctag%3Aag5zfmtoYW4tYWNhZGVteXIxCxIRQXNzZXNzbWVudEl0ZW1UYWciATAMCxIRQXNzZXNzbWVudEl0ZW1UYWcYidkZDA%2Ctag%3Aag5zfmtoYW4tYWNhZGVteXIxCxIRQXNzZXNzbWVudEl0ZW1UYWciATAMCxIRQXNzZXNzbWVudEl0ZW1UYWcY8dobDA}{Link to full list of exercises}


\section{\href{https://www.khanacademy.org/devadmin/content/items/x091b77a5}{x091b77a5}}

Phil sells hot dogs at football games. For last week's game, he bought hot dogs, buns and condiments for $\$8$ at the store, and sold the hot dogs for $\$1$ each at the game. The graph below shows the profit $y$ he made as a function of the number $x$ of hot dogs sold.

\noindent $\langle$ img $\rangle$

This week, Phil buys less hot dogs and buns, spending only $\$6$ at the store.
**What will be the graph of the profit for this week?**

[[☃ interactive-graph 1]]

Drag the two points to move the line into the correct position.

\paragraph{Hint 1}The first graph shows how the profit $\blue{y}$ varies as a function of the number of hot dogs sold $\red{x}$.
The slope of the graph is equal to $\purple{1}$, since last week Phil sold the hot dogs for $\$\purple{1}$ each. The $y$-intercept of the graph is $\green{y=-8}$,
since he invested $\$\green{8}$ for the food.

How will the graph of the profit change if Phil invests only $\$\green{6}$ for the food?

\paragraph{Hint 2}First let's **find the equation** that describes the profit of last week's sales. The equation that describes the profit $\blue{y}$ as a function of the number of hot dogs sold $\red{x}$ is    

\begin{align*}
\qquad \blue{y} 
  &=\purple{m}\red{x}+\green{b} \\
  &=\purple{1}\cdot \red{x} \green{-8}
\end{align*}  

The slope $\purple{m=1}$ corresponds to the price $\$\purple{1}$ per hot dog, and the initial value $\green{b=-8}$ corresponds to Phil's initial investment of $\$\green{8}$.

\paragraph{Hint 3}If Phil invests only $\$\green{6}$ for the food this week,
then the equation of the profit $\blue{y}$ as a function of the number of hot dogs sold $\red{x}$ will become

\begin{align*}
\qquad \blue{y} 
  &=\purple{m}\red{x}+\green{b} \\
  &=\purple{1}\cdot \red{x} \green{-6}
\end{align*}  

The initial value changed from $\green{b=-8}$ to $\green{b=-6}$ to reflect the fact that Phil spent only $\$\green{6}$ at the store this week.
The slope remained the same: $\purple{m=1}$.

\paragraph{Hint 4}The graph of the new profit function $\blue{y}=\purple{1}\cdot \red{x} \green{-6}$, passes through the point $(\red{0},\blue{-6})$ and has slope equal to $\purple{1}$:

\noindent $\langle$ img $\rangle$






\section{\href{https://www.khanacademy.org/devadmin/content/items/x1019b658}{x1019b658}}

The graph below describes the cycles of the moon 
as observed from the Earth during the first $60$ days of a year. 
The percentage of moon's surface which we see varies as a function of time. 
A *full moon* corresponds to $100\%$ visibility
and a *new moon* corresponds to $0\%$ visibility. 

\noindent $\langle$ img $\rangle$

**Complete the sentences based on the graph of the function.**

There was a new moon on the [[☃ input-number 1]]$^{\textrm{th}}$ day of the year.   
There was also a new moon on the [[☃ input-number 2]]$^{\textrm{st}}$ day of the year.  
Therefore, the period between two new moons is approximately  
[[☃ input-number 3]] days long.


Complete the sentences.

\paragraph{Hint 1}The percentage of the moon which is visible from earth varies as a function of time. The figure shows the graph of this function. We can complete the description by *reading off* the values from the graph.

\paragraph{Hint 2}A *new moon* corresponds to $\blue{M=0\%}$ visibility. We see that this occurs when $\red{t=24}$ and $\red{t=51}$. Therefore there was a new moon on the $24^{\textrm{th}}$ and $51^{\textrm{st}}$ days of the year.

\paragraph{Hint 3}To find the *period* between two new moons, we can calculate the difference between the two times when the new moon occurred: 

$\qquad$ period $=51-24=\purple{27}$ days

\paragraph{Hint 4}Note that the word *period* can be used to describe any graph that shows a repeating pattern. The period is the length of the pattern that repeats**.**





\section{\href{https://www.khanacademy.org/devadmin/content/items/x150506e4}{x150506e4}}

$\green{\textrm{Alberto}}$ and $\blue{\textrm{Bianca}}$ are having a $50\text{ km}$ race.
The illustration below shows the graph of the position  $x$ of the two runners as a function of time $t$.

\noindent $\langle$ img $\rangle$

**Complete the sentences based on the graph of the function.**

Initially, [[☃ dropdown 4]] runs faster.  
Alberto maintains a speed of [[☃ input-number 1]]$\text{ km/h}$ during the first hour of the race.  
However, after one hour he gets tired and must take a brake. Unfortunately he falls asleep!  
Meanwhile, Bianca's running speed [[☃ dropdown 2]].   
The first person to cross the finish line is [[☃ dropdown 1]].

Complete the sentences.

\paragraph{Hint 1}The graph shows the position of the two runners as a function of time. The steepness of each graph represents the speed at which the person is running. 
Let's use the graph to complete the sentences one by one.

\paragraph{Hint 2}Looking at the graph, we see that initially   $\green{\textrm{Alberto}}$'s position changes faster. The graph of $\green{\textrm{Alberto}}$'s position is steeper than that of $\blue{\textrm{Bianca}}$'s.

\paragraph{Hint 3}We can calculate $\green{\textrm{Alberto}}$'s speed by calculating the slope of the graph of $\green{\textrm{Alberto}}$'s position function as follows:

$\qquad \dfrac{\textrm{rise}}{\textrm{run}} = \dfrac{20-0\ \textrm{ [km]}}{ 1 -0 \ \textrm{ [h]}} = \dfrac{20 \textrm{ [km]}}{1 \textrm{ [h]}} = \purple{20} \ \textrm{km}/\textrm{h}$.

Alberto maintains a speed of $\purple{20}\text{ km/h}$ during the first hour of the race.

\paragraph{Hint 4}Let's now look at the graph of $\blue{\textrm{Bianca}}$'s position as a function of time. She starts out running *really* slowly, but then she picks up speed with time. 
The steepness of the graph of $\blue{\textrm{Bianca}}$'s position keeps **increasing**.

\paragraph{Hint 5}The first person to cross the finish line is $\blue{\textrm{Bianca}}$. We know this because her position graph is the first to reach the finish line at $d=50\text{ km}$.

\paragraph{Hint 6}The correct way to complete the sentences is as follows.

Initially, **Alberto** runs faster. Alberto maintains a speed of $\mathbf{20}\text{ km/h}$ during the first hour of the race. Meanwhile, Bianca's running speed **is increasing**. 
The first person to cross the finish line is **Bianca**.





\section{\href{https://www.khanacademy.org/devadmin/content/items/x160049eb}{x160049eb}}

Your friend describes to you the graph of a linear function over the phone. She says the $y$-intercept of the graph is equal to $6$ and the slope of the graph is equal to $-2$.

**Draw the graph of this function.**  

[[☃ interactive-graph 1]]

Drag the two points to move the line into the correct position. 

\paragraph{Hint 1}Let's draw the graph of the function which corresponds to the verbal description.

\paragraph{Hint 2}The $y$-intercept corresponds to the value of the function when $\red{x=0}$. Since your friend told you the $y$-intercept of the graph is $\blue{y=6}$, you know the graph of the function passes through the point $(\red{0},\blue{6})$.

Drag one of the points on the graph placing it at the coordinates $(0,6)$.

\paragraph{Hint 3}Since your friend told you the slope of the function is $\purple{-2}$, you must position the second point on the graph to produce a line which decreases by two vertical steps for each horizontal step.

The graph of the function that corresponds to your friend's verbal description is  
\noindent $\langle$ img $\rangle$

\paragraph{Hint 4}Note the linear equation that corresponds to this graph is
$\blue{y}  = \purple{-2}\cdot\red{x}  + \green{6}$.

\paragraph{Hint 5}





\section{\href{https://www.khanacademy.org/devadmin/content/items/x16750bc5}{x16750bc5}}

The maximum speed that a sailboat can reach depends on the size of the boat. The graph below shows the maximum speed $v$ that a sailboat can reach as a function of its length $\ell$.

\noindent $\langle$ img $\rangle$

**Complete the sentences based on the graph of the function.**

The longer the sailboat is, the [[☃ dropdown 1]] it can go.   
For a boat to reach the speed of $10$ kilometers per hour, it needs to be at least [[☃ input-number 1]] feet long.   
The maximum velocity that an $8$ foot boat can reach is [[☃ input-number 2]] kilometers per hour.  
A $16$ foot sailboat can be [[☃ dropdown 2]] faster than a $4$ foot sailboat. 

Complete the sentences.

\paragraph{Hint 1}The illustration shows the graph of the function that  describes how the top speed of a boat (in $\text{km/h}$) depends on the length of the boat (in $\text{ft}$). We can complete the description by *reading off* the values from the graph.

\paragraph{Hint 2}The maximum velocity of the boat increases with size. The longer the sailboat is, the faster it can go.

\paragraph{Hint 3}Looking at the graph, we see that only boats which are $\red{\ell=16}\text{ ft}$ or longer can reach the velocity  $\blue{v=10}\text{ km/h}$.

\paragraph{Hint 4}To complete the third sentence, we must read off the value of the function when $\red{\ell=8}$. The graph passes through the point $(\red{8},\blue{7.1})$,
which means that the maximum speed for a $\red{8}\text{ ft}$ boat is $\blue{v=7.1}\text{ km/h}$.

\paragraph{Hint 5}Observe that the graph of the function passes through the points $(\red{16},\blue{10})$ and $(\red{4},\blue{5})$. This means the maximum speed of a $\red{\ell=16}\text{ ft}$ boat is $\blue{v=10}\text{ km/h}$ and 
the maximum speed of a $\red{\ell=4}\text{ ft}$ boat is $\blue{v=5}\text{ km/h}$. 

Therefore, the speed of a $16$ foot boat is **2 times** faster than the speed of a $4$ foot boat.

\paragraph{Hint 6}The correct way to complete the sentences is as follows.

The longer the sailboat is, the **faster** it can go.   
For a boat to reach the speed of $10$ kilometers per hour, it needs to be at least $\mathbf{16}$ feet long.   
The maximum velocity that an $8$ foot boat can reach is $\mathbf{7.1}$ kilometers per hour.  
A $16$ foot sailboat can be $\mathbf{2}$ **times** faster than a $4$ foot sailboat. 





\section{\href{https://www.khanacademy.org/devadmin/content/items/x45bd9418}{x45bd9418}}

Ron borrowed $\$200$ from his friend and promised to return the money by paying back $\$20$ each week. Assume $x$ represents the time in weeks, and $y$ represents the amount of money left to pay.

**Draw the graph of $y$ as a function of $x$.**

[[☃ interactive-graph 1]]

Drag the two points to move the line into the correct position. 

\paragraph{Hint 1}We’re looking for the graph of the function which describes the amount of money $\blue{y}$ remaining  for Ron to pay back to his friend as a function of the time $\red{x}$ measured in weeks.

\paragraph{Hint 2}The initial value of Ron's debt is $\$\green{200}$. This is the amount Ron has to pay back. Ron's debt $\blue{y}$ decreases by $\$\purple{20}$ each week.
Ron's debt is described by the following linear equation:

 \begin{align*}
\quad \blue{y} 
 &= \purple{m}\cdot\red{x}  + \green{b} \\[1mm]
 &= \purple{-20}\cdot\red{x}  + \green{200}.
\end{align*}

The initial value of the loan is $\green{b=200}$. This is the amount Ron owes to his friend when $\red{x=0}$.
The rate of change of the function is $\purple{m=-20}$ because each week Ron pays back $\$\purple{20}$ to his friend.

\paragraph{Hint 3}The graph of the function $\blue{y}=\purple{m}\cdot\red{x}+\green{b}$ is a line with $y$-intercept equal to $\green{b}$ and slope equal to $\purple{m}$.

Therefore, the graph of Ron's debt $\blue{y} = \green{200}  \purple{-20}\red{x}$ is a line which passes through the point $(\red{0},\blue{200})$ and has slope equal to $\purple{-20}$:   
\noindent $\langle$ img $\rangle$





\section{\href{https://www.khanacademy.org/devadmin/content/items/x572138ad}{x572138ad}}

The illustration below shows the graph of $y$ as a function of $x$.

\noindent $\langle$ img $\rangle$

**Complete the sentences based on the graph.**

Initially, as $x$ increases, $y$ [[☃ dropdown 1]].  
Afterward, the slope of the graph of the function is equal to [[☃ input-number 1]] for all $x$ between $x=3$ and $x=5$.  
The slope of the graph is equal to [[☃ input-number 2]] for $x$ between $x=5$ and $x=9$.  
The greatest value of $y$ is $y=\:$[[☃ input-number 3]] and it occurs when $x=\:$[[☃ input-number 4]].

Complete the sentences.

\paragraph{Hint 1}Let's use the graph to complete the sentences one by one.

\paragraph{Hint 2}Initially, the graph of the function is flat. The quantity $\blue{y}$ remains constant as $\red{x}$ increases from $\red{x=0}$ to $\red{x=3}$. 

The linear equation $\blue{y}=\purple{m}\red{x}+\green{b}$ that corresponds to the function in that region is $\blue{y}=\purple{0}\cdot\red{x}+\green{0}$. The slope is $\purple{m=0}$ (since the graph is a horizontal line) and the initial value is $\green{b=0}$.

\paragraph{Hint 3}Starting at $\red{x=3}$ the graph of the function starts to increase. By counting the squares of the grid, we can calculate the slope of the graph:

$\qquad \dfrac{\blue{\textrm{rise}}}{\red{\textrm{run}}} = \dfrac{\blue{4} - \blue{0}}{ \red{5} - \red{3}} =\dfrac{4}{2} = 2$.

The slope of the graph is $\purple{2}$ for $\red{x}$ between $\red{3}$ and $\red{5}$.

\paragraph{Hint 4}After $\red{x=5}$, $\blue{y}$ starts to decrease as $\red{x}$ increases. The slope of the graph between $\red{x=5}$ and $\red{x=9}$ is equal to:

$\qquad \dfrac{\blue{\textrm{rise}}}{\red{\textrm{run}}} = \dfrac{\blue{0} - \blue{4}}{ \red{9} - \red{5}} =\dfrac{-4}{4} = -1$.

The slope of the graph is equal to $\purple{-1}$ for $\red{x}$ between $\red{5}$ and $\red{9}$.

\paragraph{Hint 5}The maximum value of $\blue{y}$ in the graph occurs at the "peak" of the bump in the graph. The coordinates of this "peak" are $(\red{5},\blue{4})$. Therefore, the maximum value is $\blue{y=4}$ and it occurs when $\red{x=5}$.

\paragraph{Hint 6}After completing the sentences we obtain the following description of the graph.

Initially, as $x$ increases, $y$ **stays constant**.  
Afterward, the slope of the graph of the function is equal to $\mathbf{2}$ for all $x$ between $x=3$ and $x=5$.  
The slope of the graph is equal to $\mathbf{-1}$ for $x$ between $x=5$ and $x=9$.  
The greatest value of $y$ is $y=\mathbf{4}$ and it occurs when $x=\mathbf{5}$.





\section{\href{https://www.khanacademy.org/devadmin/content/items/x58aec930}{x58aec930}}

Miriam overhears her teacher talking about a function described by a linear equation.
The rate of change of this function is $3$ and its initial value is $-6$. 

**What is the graph of this function?**

[[☃ interactive-graph 1]]

Drag the two points to move the line into the correct position. 

\paragraph{Hint 1}We know the function is described by a linear equation of the form:   

$\quad \blue{y} = \purple{m}\cdot\red{x} + \green{b}$.

The number $\purple{m}$ is called the rate of change of the function. The number $\green{b}$ is the initial value of the function.



\paragraph{Hint 2}Miriam heard the teacher say the rate of change of this function is $\purple{m=3}$ and its initial value is $\green{b=-6}$, therefore the equation which describes this function is:

$\quad \blue{y}  = \purple{3}\cdot\red{x}   \green{-6}$.


\paragraph{Hint 3}The graph of the function $\blue{y}=\purple{m}\cdot\red{x}+\green{b}$ is a line with $y$-intercept equal to $\green{b}$ and slope equal to $\purple{m}$.

Therefore, the graph of the function Miriam's teacher is talking about is a line which passes through the point $(\red{0},\blue{-6})$ and has slope equal to $\purple{3}$:  
\noindent $\langle$ img $\rangle$





\section{\href{https://www.khanacademy.org/devadmin/content/items/x59b75739}{x59b75739}}

The illustration below shows the graph of $y$ as a function of $x$.

\noindent $\langle$ img $\rangle$

**Complete the sentences based on the graph of the function.**

This is the graph of a [[☃ dropdown 3]] function.   
The $y$-intercept of the graph is $y=\:$[[☃ input-number 1]].  
The $x$-intercepts of the graph are at $x=\:$[[☃ input-number 2]] and $x=\:$[[☃ input-number 5]].  
The greatest value of $y$ is $y=\:$[[☃ input-number 3]] and it occurs when $x=\:$[[☃ input-number 4]].  
For $x$ between $x=0$ and $x=6$,  $y\:$[[☃ dropdown 1]]$\:0$.


Complete the sentences.

\paragraph{Hint 1}Let's use the graph to complete the sentences one by one.

\paragraph{Hint 2}A function which has a constant rate of change produces a graph that is a line and we say the function is *linear*. Otherwise, if the rate of change of the function is not constant, the graph will not be a line, and we say the function is *nonlinear*.

The graph shown is not a line. Therefore, we are looking at the graph of a **nonlinear** function.

\paragraph{Hint 3}The $y$-intercept of the graph corresponds to the value of the function when $\red{x=0}$.
The $y$-intercept of the graph is at $\blue{y=0}$.

\paragraph{Hint 4}The $x$-intercepts of the graph correspond to the values of $\red{x}$ for which $\blue{y=0}$. The $x$-intercepts of this graph are at $\red{x=0}$ and $\red{x=6}$.

\paragraph{Hint 5}The maximum value of $\blue{y}$ in the graph occurs at the top of the bump. The coordinates of the top of the bump are $(\red{3},\blue{9})$. Therefore, the maximum value is $\blue{y=9}$ and it occurs when $\red{x=3}$.

\paragraph{Hint 6}The graph of the function is above the $x$-axis
for values of $\red{x}$ between $\red{x=0}$ and $\red{x=6}$. For these values of $\red{x}$,  $y\geq 0$.

\paragraph{Hint 7}We can now complete the sentences.

This is the graph of a **nonlinear** function.   
The $y$-intercept of the graph is $y=\mathbf{0}$.  
The $x$-intercepts of the graph are at $x=\mathbf{0}$ and $x=\mathbf{6}$.  
The greatest value of $y$ is $y=\mathbf{9}$ and it occurs when $x=\mathbf{3}$.  
For $x$ between $x=0$ and $x=6$,  $\mathbf{y\geq 0}$.






\section{\href{https://www.khanacademy.org/devadmin/content/items/x6ad99bbe}{x6ad99bbe}}

The variable $y$ depends on the variable $x$. When $x$ increases by one unit, $y$ increases by $3$ units. 
Also, when $x=1$, $y=5$.

**Draw the graph of $y$ as a function of $x$.**

[[☃ interactive-graph 1]]

Drag the two points to move the line into the correct position. 

\paragraph{Hint 1}We want to draw the graph of the function which corresponds to the verbal description. Let's see how to find the graph of the function given the information provided.

\paragraph{Hint 2}We are told that $\blue{y=5}$ when $\red{x=1}$.
In other words, the graph of the function passes through the point $(\red{1},\blue{5})$.

You can drag one of the points on the graph to the coordinates $(\red{1},\blue{5})$ since we know the graph passes through there.

\paragraph{Hint 3}We are also told that when $\red{x}$ increases by one unit, $\blue{y}$ increases by $\purple{3}$ units. This is another way of saying the slope of the graph is equal to $\purple{3}$.

\paragraph{Hint 4}**We now have enough information to draw the graph of $\blue{y}$ as a function of $\red{x}$. Drag the second point to produce a line with slope equal to $\purple{3}$. You can do this by placing the second point at $(2,8)$, which is $1$ unit to the right and $\purple{3}$ units higher than the point $(1,5)$. The result will look like this:**

\noindent $\langle$ img $\rangle$

\paragraph{Hint 5}Note that this is the graph of the function $\blue{y}=\purple{3}\cdot \red{x} + \green{2}$**.**





\section{\href{https://www.khanacademy.org/devadmin/content/items/x797d9dd8}{x797d9dd8}}

Korey is organizing a music concert. She plans to invest $\$2000$ to rent the venue and to pay the musicians and then sell the tickets for $\$20$ each. The graph below shows the profit $y$ she will make from the concert as a function of the number of tickets sold $x$.


\noindent $\langle$ img $\rangle$

If instead Korey decides to sell the tickets for $\$40$ each, **what will be the new graph of Korey's profit?**

[[☃ interactive-graph 1]]

Drag the two points to move the line into the correct position.

\paragraph{Hint 1}The first graph shows how the profit $\blue{y}$ varies as a function of the number of tickets sold $\red{x}$.
The slope of the graph is equal to $\purple{20}$, since Korey plans to sell the tickets for $\$\purple{20}$. The $y$-intercept of the graph is $\blue{y}=\green{-2000}$,
since she plans to invest $\$\green{2000}$ to organize the concert.

How will the graph of the profit change if Korey decides to sell the tickets for $\$\purple{40}$ each?

\paragraph{Hint 2}Let's **find the equation** that describes Korey's profit when she invests $\$\green{2000}$ and sells the tickets for $\$\purple{20}$. 
In that case, the equation that describes the profit $\blue{y}$ as a function of the number of tickets sold $\red{x}$ is    

\begin{align*}
\qquad \blue{y} 
  &=\purple{m}\red{x}+\green{b} \\
  &=\purple{20}\cdot \red{x} \green{-2000}
\end{align*}  

The rate of change $\purple{m=20}$ corresponds to the price $\$\purple{20}$ per ticket, and the initial value $\green{b=-2000}$ corresponds to Korey's initial investment of $\$\green{2000}$.

\paragraph{Hint 3}If instead Korey sells the tickets for $\$\purple{40}$ each, then the equation of the profit $\blue{y}$ as a function of the number of tickets sold $\red{x}$ will become

\begin{align*}
\qquad \blue{y} 
  &=\purple{m}\red{x}+\green{b} \\
  &=\purple{40}\cdot \red{x} \green{-2000}
\end{align*}  

The slope changed from $\purple{m=20}$ to $\purple{m=40}$ since Korey now sells the tickets for $\$\purple{40}$.

\paragraph{Hint 4}The graph of the new profit function $\blue{y}=\purple{40}\cdot \red{x} \green{-2000}$, passes through the point $(\red{0},\blue{-2000})$ and has slope equal to $\purple{40}$:

\noindent $\langle$ img $\rangle$

Note the slope of the graph changed but the $y$-intercept of the graph remained the same.





\section{\href{https://www.khanacademy.org/devadmin/content/items/x81f41d07}{x81f41d07}}

Jimmy will be selling hot dogs at the football game. He bought hot dogs, buns and condiments for $\$8$, and plans to sell the hot dogs for $\$1$ each at the game. The graph below shows the profit $y$ he will make as a function of the number $x$ of hot dogs sold.

\noindent $\langle$ img $\rangle$

**If Jimmy sells the hot dogs for $\$2$ instead, what will be the new graph of the profit?**

[[☃ interactive-graph 1]]

Drag the two points to move the line into the correct position.

\paragraph{Hint 1}The first graph shows how the profit $\blue{y}$ varies as a function of the number of hot dogs sold $\red{x}$.
The slope of the graph is equal to $\purple{1}$, since Jimmy sells the hot dogs for $\$\purple{1}$ each. The $y$-intercept of the graph is $\green{y=-8}$.

How will the graph of the profit change if Jimmy sells the hot dogs for $\$\purple{2}$ each?

\paragraph{Hint 2}First let's **find the equation** that describes the profit. The equation that describes the profit $\blue{y}$ as a function of the number of hot dogs sold $\red{x}$ is    

\begin{align*}
\qquad \blue{y} 
  &=\purple{m}\red{x}+\green{b} \\
  &=\purple{1}\cdot \red{x} \green{-8}
\end{align*}  

The slope $\purple{m=1}$ corresponds to the price $\$\purple{1}$ per hot dog, and the $y$-intercept $\green{b=-8}$ corresponds to Jimmy's initial investment of $\$\green{8}$.

\paragraph{Hint 3}If Jimmy sells the hot dogs for $\$\purple{2}$ each, then the equation of the profit $\blue{y}$ as a function of the number of hot dogs sold $\red{x}$ will become

\begin{align*}
\qquad \blue{y} 
  &=\purple{m}\red{x}+\green{b} \\
  &=\purple{2}\cdot \red{x} \green{-8}
\end{align*}  

The slope changed to $\purple{m=2}$ since Jimmy sells the hot dogs for $\$\purple{2}$ each. 
Jimmy's initial investment is still $\$\green{8}$,
so the $y$-intercept remains $\green{b=-8}$.

\paragraph{Hint 4}The graph of the new profit function $\blue{y}=\purple{2}\cdot \red{x} \green{-8}$, passes through the point $(\red{0},\blue{-8})$ and has slope equal to $\purple{2}$:

\noindent $\langle$ img $\rangle$






\section{\href{https://www.khanacademy.org/devadmin/content/items/x85e40c41}{x85e40c41}}

The illustration below shows the graph of $y$ as a function of $x$.

\noindent $\langle$ img $\rangle$

**Complete the sentences based on the graph of the function.**

Initially, as $x$ increases, $y$  [[☃ dropdown 4]].    
The slope of the graph is equal to [[☃ input-number 1]] for all $x$ between $x=0$ and $x=3$.  
Starting at $x=3$,  $y$ [[☃ dropdown 2]] as $x$ increases.  
The slope of the graph is equal to [[☃ input-number 2]] for $x$ between $x=3$ and $x=5$.  
For $x$ between $x=0$ and $x=4$,  $y$ [[☃ dropdown 3]] $0$.   
For $x$ between $x=4$ and $x=8$, $y$ [[☃ dropdown 1]] $0$.

Complete the sentences.

\paragraph{Hint 1}Let's use the graph to complete the sentences one by one.

\paragraph{Hint 2}Initially, the graph of the function is decreasing. The quantity $\blue{y}$ decreases as $\red{x}$ increases from $\red{x=0}$ to $\red{x=3}$. 

By counting the squares of the grid, we can calculate the slope of the graph:

$\qquad \dfrac{\blue{\textrm{rise}}}{\red{\textrm{run}}} = \dfrac{\blue{-3} - \blue{0}}{ \red{3} - \red{0}} =\dfrac{-3}{3} = \purple{-1}$.

So the slope of the line is $\purple{m=-1}$.

\paragraph{Hint 3}Starting at $\red{x=3}$ the graph of the function starts to increase. By counting the squares of the grid again, we can calculate the slope of the graph:

$\qquad \dfrac{\blue{\textrm{rise}}}{\red{\textrm{run}}} = \dfrac{\blue{3} - (\blue{-3})}{ \red{5} - \red{3}} =\dfrac{6}{2} = \purple{3}$.

The slope of the graph is $\purple{3}$ for $\red{x}$ between $\red{3}$ and $\red{5}$.

\paragraph{Hint 4}Now we're done completing the sentences which talk about the $\purple{\textrm{slope}}$ of the function. Next we have to complete the sentences that describe the $\blue{\textrm{values}}$ of the function. For what values of the input $\red{x}$ is $\blue{y}$ positive and for what values of $\red{x}$ is $\blue{y}$ negative?

The value of $\blue{y}$ is positive whenever the graph of the function is above the $x$-axis. The value of $\blue{y}$ is negative whenever the graph is below the $x$-axis.
 

\paragraph{Hint 5}The graph of the function is below the $x$-axis for $\red{x}$ between $\red{x=0}$ and $\red{x=4}$, which means the value of $\blue{y}$ is negative.
For $\red{x}$ between $\red{x=0}$ and $\red{x=4}$, $\blue{y} \leq 0$.

\paragraph{Hint 6}The graph of the function is above the $x$-axis for $\red{x}$ between $\red{x=4}$ and $\red{x=8}$, which means the value of $\blue{y}$ is positive.
For $\red{x}$ between $\red{x=4}$ and $\red{x=8}$, $\blue{y} \geq 0$.

\paragraph{Hint 7}We now know how to complete the sentences.

Initially, as $x$ increases, $y$  **decreases**.    
The slope of the graph is equal to $\mathbf{-1}$ for all $x$ between $x=0$ and $x=3$.  
Starting at $x=3$,  $y$ **increases** as $x$ increases.  
The slope of the graph is equal to $\mathbf{3}$ for $x$ between $x=3$ and $x=5$.  
For $x$ between $x=0$ and $x=4$,  $\mathbf{y \leq 0}$.   
For $x$ between $x=4$ and $x=8$, $\mathbf{y \geq 0}$.





\section{\href{https://www.khanacademy.org/devadmin/content/items/x906998f8}{x906998f8}}

Jimmy will be selling hot dogs at the football game. He bought hot dogs, buns and condiments for $\$8$, and now wants to calculate how many hot dogs he has to sell to make a profit. He makes a graph of the profit $P\:$ he will make as a function of the number $n$ of hot dogs sold.

\noindent $\langle$ img $\rangle$

**Complete the sentences based on the graph of the function.**

As Jimmy sells more hot dogs, 
his profit will [[☃ dropdown 1]].  
Jimmy sells the hot dogs for $\$$[[☃ input-number 1]] each.   
Jimmy needs to sell [[☃ input-number 2]] hot dogs to recover the money he invested.  
If Jimmy sells $15$ hot dogs, he will make $\$$[[☃ input-number 3]] in profit.  
To make $\$12$ in profit, he would have to sell [[☃ input-number 4]] hot dogs.

Complete the sentences.

\paragraph{Hint 1}The graph shows how the profit $\blue{P}\:$ varies as a function of the number of hot dogs sold $\red{n}$. We can use the graph of the function to complete the sentences.

\paragraph{Hint 2}The graph of the function is increasing. As $\red{n}$ increases, the profit $\blue{P}\:$ will **increase**.

The slope of the graph is equal to $\purple{1}$.
For each hot dog sold, the profit increases by $\$\purple{1}$. This means Jimmy is selling the hot dogs for $\$\purple{1}$ each.

\paragraph{Hint 3}To complete the remaining three sentences, we can *read off* the appropriate values from the graph of the function. Observe that the graph of the function passes through the points $(\red{8},\blue{0})$, $(\red{15},\blue{7})$, and $(\red{20},\blue{12})$. Therefore, Jimmy needs to sell $\red{8}$ hot dogs to recover his investment (to break even), $\red{15}$ hot dogs to make a profit of $\blue{P=\$7}$, and $\red{20}$ hot dogs to make a profit of $\blue{P=\$12}$.

\paragraph{Hint 4}Let's check the answers by using the **equation** which corresponds to the function. Since the graph shows a linear relationship between the profit $\blue{P}\:$ and number of hot dogs sold, the equation which describes the profit must be of the form $\blue{P}=\purple{m}\red{n}+\green{b}$, where $\purple{m}$ is the slope and $\green{b}$ is the initial value. The equation that describes the profit $\blue{P}$ as a function of the number of hot dogs sold $\red{n}$ is    
\begin{align*}
\qquad \blue{P} 
  &=\purple{m}\red{n}+\green{b} \\
  &=\purple{1}\cdot \red{n} \green{-8}.
\end{align*}  
The slope $\purple{m=1}$ corresponds to the price per hot dog, and the $y$-intercept $\green{b=-8}$ corresponds to Jimmy's initial investment.

The slope is $\purple{m=1} \geq 0$, therefore $P\:$ increases with $\red{n}$.
To find how many hot dogs Jimmy needs to sell to recover the money he invested, we must solve for $\red{n}$ the equation $\blue{P}=0$:  
\begin{align*}
 \blue{P} &= 0 \\
 \ \  \purple{1}\cdot \red{n} \green{-8} & = 0 \\ 
 \red{n} &= 8.
\end{align*}

Similarly, to find the number of sales required to make $\$\blue{12}$ in profit, we must find $\red{n}$ when $\blue{P}=12$:  
\begin{align*}
 \blue{P} &= 12 \\
 \ \ \purple{1}\cdot \red{n} \green{-8} & = 12 \\ 
 \red{n} & = 12 + 8\\ 
 \red{n} &= 20.
\end{align*}

To find the profit Jimmy will make if he sells $\red{15}$ hot dogs, he plug the value $\red{n=15}$ into the equation for $\blue{P}\:$ to obtain  
\begin{align*}
\quad \blue{P} 
  &=\purple{m}\red{n}+\green{b} \\
  &= \purple{1}\cdot \red{15} \green{-8} \\ 
   &= 7.
\end{align*}

\paragraph{Hint 5}The correct way to complete the sentences is as follows.

As Jimmy sells more hot dogs, 
his profit will **increase**.  
Jimmy sells the hot dogs for $\$\mathbf{1}$ each.   
Jimmy needs to sell $\mathbf{8}$ hot dogs to recover the money he invested.  
If Jimmy sells $15$ hot dogs, he will make $\$\mathbf{7}$ in profit.  
To make $\$12$ in profit, he would have to sell $\mathbf{20}$ hot dogs.





\section{\href{https://www.khanacademy.org/devadmin/content/items/x93a49507}{x93a49507}}

David wants to rent a bicycle for a couple of hours to explore the city. 
The price of the bike rental $P$ depends on the time the bike was out $t$.
The price consists of a base charge of $\$8$ and a variable hourly cost.
The graph of the function is shown below.

\noindent $\langle$ img $\rangle$

**Complete the sentences based on the graph of the function.**

The hourly charge is $\$$[[☃ input-number 1]] per hour for the first $3$ hours.  
The rate then drops to $\$$[[☃ input-number 2]] per hour until the end of the $6^{\textrm{th}}$ hour.   
The hourly rate drops further to $\$$[[☃ input-number 3]]  per hour between $6^{\textrm{th}}$ and the $10^{\textrm{th}}$ hour.  
The maximum price of the bike rental is [[☃ input-number 4]] dollars.




Complete the sentences.

\paragraph{Hint 1}The slope of the graph corresponds to the hourly rate for the bike rental.

\paragraph{Hint 2}During the first three hours of the bike rental, the price increases by $\blue{\$4}$ each hour.

\paragraph{Hint 3}Between third hour and the sixth hour, the slope of the graph is $\purple{2}$, which means the hourly rate of the bike rental is $\$\purple{2}$ per hour.

\paragraph{Hint 4}For the hours between the $6^{\textrm{th}}$ and $10^{\textrm{th}}$ hour, the price is $\$\red{1}$ per hour.


\paragraph{Hint 5}After the $10^{\textrm{th}}$ hour, the price $P$ stops increasing. The maximum price of the bike rental is $\$\pink{30}$.

\paragraph{Hint 6}The graph of the price function is horizontal after $t=10$ hours.  The $\green{\textrm{slope}}$ of the graph is $\green{0}$. 

This means the price function is constant. 
After $t=10$ hours, the equation which describes the price is:

$\qquad P=\green{0}\cdot t + 30 = 30$

This is the equation of a line with a slope $\green{m=0}$.

\paragraph{Hint 7}The correct way to complete the sentences is as follows.

The hourly charge is $\$\mathbf{4}$ per hour for the first $3$ hours.  
The rate then drops to $\$\mathbf{2}$ per hour until the end of the $6^{\textrm{th}}$ hour.   
The hourly rate drops further to $\$\mathbf{1}$ per hour between $6^{\textrm{th}}$ and the $10^{\textrm{th}}$ hour.  
The maximum price of the bike rental is $\mathbf{30}$ dollars**.**





\section{\href{https://www.khanacademy.org/devadmin/content/items/xa0f88ec4}{xa0f88ec4}}

You are downloading a file from the Internet.
Your download rate is equal to $10$ megabytes per minute.  Assume $x$ represents the time in minutes, and $y$ represents the amount of downloaded data in megabytes.

**Draw the graph that represents the amount of downloaded data as a function of time.**

[[☃ interactive-graph 1]]

Drag the two points to move the line into the correct position. 

\paragraph{Hint 1}We’re looking for the graph of the function which describes how the amount of downloaded data $\blue{y}$ depends on the time $\red{x}$. 

\paragraph{Hint 2}The amount of downloaded data $\blue{y}$ is *proportional* to the time $\red{x}$. We can think of $\blue{y}$ as a function of $\red{x}$. The equation of this function is

 \begin{align*}
\quad \blue{y} 
 &= \purple{m}\cdot\red{x} \\[1mm]
 &= \purple{10}\cdot\red{x} 
\end{align*}

The constant of proportionality is $\purple{m=10}$ megabytes per minute, since this is the download rate. Indeed, a download speed of $\purple{10}$ megabytes per minute means the download data increases by $\purple{10}$ megabytes each minute.

\paragraph{Hint 3}The graph of the function $ \blue{y} = \purple{10}\red{x}$ is a line which passes through the origin and has slope equal to $\purple{10}$:   
\noindent $\langle$ img $\rangle$

\paragraph{Hint 4}





\section{\href{https://www.khanacademy.org/devadmin/content/items/xa79c5488}{xa79c5488}}

Jeff works as a waiter at a fancy restaurant. He receives a base amount of $\$70$ each day, plus approximately $\$10$ of tip for each client he serves. Assume $x$ represents the number of clients Jeff will serve in a day and $y$ represents his daily salary.

**Draw the graph which represents Jeff's daily salary as a function of the number of clients he serves.**

[[☃ interactive-graph 1]]

Drag the two points to move the line into the correct position. 

\paragraph{Hint 1}We’re looking for the graph of the function which describes how Jeff's daily salary $\blue{y}$ depends on the number of clients he serves $\red{x}$.

\paragraph{Hint 2}We know his salary $\blue{y}$ increases by $\$\purple{10}$ for each client he serves. Also, we know that he receives a base daily salary of $\$\green{70}$. 
Jeff's daily salary $\blue{y}$ as a function of the number of clients he serves $\red{x}$ is described by the following linear equation:

 \begin{align*}
\quad \blue{y} 
 &= \purple{m}\cdot\red{x}  + \green{b} \\[1mm]
 &= \purple{10}\cdot\red{x}  + \green{70}.
\end{align*}

The rate of change is $\purple{m=10}$ because this is how much tip he makes per client. If $\red{x}$ increases by one, his salary $\blue{y}$ will increase by $\$\purple{10}$. 

The initial value is $\green{b=70}$. This is the base amount Jeff earns even when he serves $\red{x=0}$ clients.

\paragraph{Hint 3}The graph of the function $\blue{y}=\purple{m}\cdot\red{x}+\green{b}$ is a line with $y$-intercept equal to $\green{b}$ and slope equal to $\purple{m}$.

Therefore, the graph of Jeff's daily salary $ \blue{y} = \purple{10}\red{x} + \green{70}$ is a line which passes through the point $(\red{0},\blue{70})$ and has slope equal to $\purple{10}$:   

\noindent $\langle$ img $\rangle$

\paragraph{Hint 4}





\section{\href{https://www.khanacademy.org/devadmin/content/items/xadd760c2}{xadd760c2}}

Korey is organizing a music concert. She plans to invest $\$2000$ to rent the venue and to pay the musicians and then sell the tickets for $\$20$ each. The graph below shows the profit $y$ she will make from the concert as a function of the number of tickets sold $x$.

\noindent $\langle$ img $\rangle$

If instead Korey decides to invest only $\$1200$ for the concert and sells the tickets for $\$40$ each,
**what will be the new graph of Korey's profit?**

[[☃ interactive-graph 1]]

Drag the two points to move the line into the correct position.

\paragraph{Hint 1}The first graph shows how the profit $\blue{y}$ varies as a function of the number of tickets sold $\red{x}$.
The slope of the graph is equal to $\purple{20}$, since Korey plans to sell the tickets for $\$\purple{20}$. The $y$-intercept of the graph is $\blue{y}=\green{-2000}$,
since she plans to invest $\$\green{2000}$ to organize the concert.

How will the graph of the profit change if Korey invests $\$\green{1200}$ and sells the tickets for $\$\purple{40}$ each?

\paragraph{Hint 2}Let's **find the equation** that describes Korey's profit when she invests $\$\green{2000}$ and sells the tickets for $\$\purple{20}$. 
In that case, the equation that describes the profit $\blue{y}$ as a function of the number of tickets sold $\red{x}$ is    

\begin{align*}
\qquad \blue{y} 
  &=\purple{m}\red{x}+\green{b} \\
  &=\purple{20}\cdot \red{x} \green{-2000}
\end{align*}  

The rate of change $\purple{m=20}$ corresponds to the price $\$\purple{20}$ per ticket, and the initial value $\green{b=-2000}$ corresponds to Korey's initial investment of $\$\green{2000}$.

\paragraph{Hint 3}If instead Korey invests $\$\green{1200}$ for the concert and sells the tickets for $\$\purple{40}$ each,
then the equation of the profit $\blue{y}$ as a function of the number of tickets sold $\red{x}$ will become

\begin{align*}
\qquad \blue{y} 
  &=\purple{m}\red{x}+\green{b} \\
  &=\purple{40}\cdot \red{x} \green{-1200}
\end{align*}  

The initial value changed from $\green{b=-2000}$ to $\green{b=-1200}$ to reflect Korey's smaller up-front investment.
The slope changed to $\purple{m=40}$ since Korey now sells the tickets for $\$\purple{40}$.

\paragraph{Hint 4}The graph of the new profit function $\blue{y}=\purple{40}\cdot \red{x} \green{-1200}$, passes through the point $(\red{0},\blue{-1200})$ and has slope equal to $\purple{40}$:

\noindent $\langle$ img $\rangle$






\section{\href{https://www.khanacademy.org/devadmin/content/items/xb754f5a4}{xb754f5a4}}

The illustration below shows the graph of $y$ as a function of $x$.

\noindent $\langle$ img $\rangle$

**Complete the sentences based on the graph of the function.**

As $x$ increases, $y$ [[☃ dropdown 1]].  
The rate of change of $y$ as $x$ changes is [[☃ dropdown 2]],
therefore the function is [[☃ dropdown 3]].  
For all values of $x$, $y\:$[[☃ dropdown 4]]$\:0$.    
The $y$-intercept of the function is $y=\:$[[☃ input-number 1]].   
When $x=1$, $y=\:$[[☃ input-number 2]].

Complete the sentences.

\paragraph{Hint 1}Let's use the graph to complete the sentences one by one.

\paragraph{Hint 2}Looking at the graph of the function, we see that $\blue{y}$ always **decreases** as $\red{x}$ increases.

\paragraph{Hint 3}A function which has a constant rate of change produces a graph that is a line and we say the function is *linear*. Otherwise, if the rate of change of the function is not constant, the graph will not be a line, and we say the function is *nonlinear*.

In the graph shown, the rate of change of $\blue{y}$ as $\red{x}$ changes is **not constant**. This means the function whose graph we are viewing is **nonlinear**.

\paragraph{Hint 4}Observe that the graph of the function lies entirely above the $x$-axis. This means the values of $\blue{y}$ are always positive.
For all values of $\red{x}$, $\blue{y}\geq 0$.   

\paragraph{Hint 5}The $y$-intercept of the graph corresponds to the value of the function when $\red{x=0}$.
The $y$-intercept of the function is $\blue{y=8}$.

\paragraph{Hint 6}To complete the last sentence, we can read off the value of the function for $\red{x=1}$. 
When $\red{x=1}$, $\blue{y=4}$.

\paragraph{Hint 7}We can now complete the sentences.

As $x$ increases, $y$ **decreases**.  
The rate of change of $y$ as $x$ changes is **not constant**,
therefore the function is **nonlinear**.  
For all values of $x$, $\mathbf{y \geq 0}$.    
The $y$-intercept of the function is $y=\mathbf{8}$.
When $x=1$, $y=\mathbf{4}$.






\section{\href{https://www.khanacademy.org/devadmin/content/items/xc78ba6a5}{xc78ba6a5}}

The illustration below shows the graph of $y$ as a function of $x$.

\noindent $\langle$ img $\rangle$

**Complete the sentences based on the graph of the function.**

As $x$ increases, $y$ [[☃ dropdown 1]].  
The rate of change of $y$ as $x$ changes is [[☃ dropdown 2]],
therefore the function is [[☃ dropdown 3]].  
For all values of $x$, $y\:$[[☃ dropdown 4]]$\:0$.    
The $y$-intercept of the function is $y=\:$[[☃ input-number 1]].   
When $x=6$, $y=\:$[[☃ input-number 2]].

Complete the sentences.

\paragraph{Hint 1}Let's use the graph to complete the sentences one by one.

\paragraph{Hint 2}Looking at the graph of the function, we see that $\blue{y}$ always **increases** as $\red{x}$ increases.

\paragraph{Hint 3}A function which has a constant rate of change produces a graph that is a line and we say the function is *linear*. Otherwise, if the rate of change of the function is not constant, the graph will not be a line, and we say the function is *nonlinear*.

In the graph shown, the rate of change of $\blue{y}$ as $\red{x}$ changes is **not constant**. This means the function whose graph we are viewing is **nonlinear**.

\paragraph{Hint 4}Observe that the graph of the function lies entirely above the $x$-axis. This means the values of $\blue{y}$ are always positive.
For all values of $\red{x}$, $\blue{y}\geq 0$.   

\paragraph{Hint 5}The $y$-intercept of the graph corresponds to the value of the function when $\red{x=0}$.
The $y$-intercept of the function is $\blue{y=1}$.

\paragraph{Hint 6}To complete the last sentence, we can read off the value of the function for $\red{x=6}$. 
When $\red{x=6}$, $\blue{y=7}$.

\paragraph{Hint 7}We can now complete the sentences.

As $x$ increases, $y$ **increases**.  
The rate of change of $y$ as $x$ changes is **not constant**, therefore the function is **nonlinear**.  
For all values of $x$, $\mathbf{y \geq 0}$.    
The $y$-intercept of the function is $y=\mathbf{1}$.   
When $x=6$, $y=\mathbf{7}$.





\section{\href{https://www.khanacademy.org/devadmin/content/items/xcb3ee656}{xcb3ee656}}

Jessica works in sales. Her monthly salary is calculated as a base amount of $\$2000$ plus a commission of $\$100$ for each sale she closes. Assume $x$ represents the number of sales Jessica closes and $y$ represents her monthly salary.

**Draw the graph which represents Jessica's monthly salary as a function of the number of sales.**

[[☃ interactive-graph 1]]

Drag the two points to move the line into the correct position. 

\paragraph{Hint 1}We’re looking for the graph of the function which describes how Jessica's monthly salary $\blue{y}$ depends on the number of sales $\red{x}$.

\paragraph{Hint 2}We know her salary $\blue{y}$ increases by $\$\purple{100}$ for each sale she makes. Also, we know that she receives a base amount of $\$\green{2000}$. 
Jessica's monthly salary $\blue{y}$ as a function of the number of sales $\red{x}$ is described by the following linear equation:

 \begin{align*}
\quad \blue{y} 
 &= \purple{m}\cdot\red{x}  + \green{b} \\[1mm]
 &= \purple{100}\cdot\red{x}  + \green{2000}.
\end{align*}

The rate of change is $\purple{m=100}$ because this is how much she makes per sale. If $\red{x}$ increases by one, her salary $\blue{y}$ will increase by $\$\purple{100}$. 

The initial value is $\green{b=2000}$. This is the base amount Jessica earns even when she makes $\red{x=0}$ sales.

\paragraph{Hint 3}The graph of the function $\blue{y}=\purple{m}\cdot\red{x}+\green{b}$ is a line with $y$-intercept equal to $\green{b}$ and slope equal to $\purple{m}$.

Therefore, the graph of Jessica's monthly salary $ \blue{y} = \purple{100}\red{x} + \green{2000}$ is a line which passes through the point $(\red{0},\blue{2000})$ and has slope equal to $\purple{100}$:   
\noindent $\langle$ img $\rangle$

\paragraph{Hint 4}





\section{\href{https://www.khanacademy.org/devadmin/content/items/xccd47986}{xccd47986}}

Today, Sean left home at $8$AM, drove to work, and worked from $9$AM until $5$PM. On the drive back home, Sean got stuck in traffic for $2$ hours, then stopped to have dinner and finally got home at $9\!:\!30$PM. 

The graph below shows Sean's distance from home $d$ as a function of the time $t$.

\noindent $\langle$ img $\rangle$

**Complete the sentences based on the graph of the function.**

Sean's work is located at a distance [[☃ input-number 1]]$\text{ km}$ from his home.   
Sean's speed when he was driving to work was [[☃ input-number 2]]$\text{ km/h}$.  
Sean's speed between $5$PM and $7$PM was 
[[☃ input-number 3]]$\text{ km/h}$.


Complete the sentences.

\paragraph{Hint 1}The graph shows Sean's position as a function of time. The slope of the graph represents the speed at which he is moving. Let's use the graph to complete the sentences one by one.

\paragraph{Hint 2}Looking at the graph, we see that Sean's position stays constant at $\blue{d=20}\text{ km}$, between $9$AM and $5$PM. This represents the time Sean spends at work.  Therefore, his work place is located at a distance of $\blue{20}\text{ km}$ from his home.

\paragraph{Hint 3}We can calculate the slope of the graph 
of Sean's position function between $\red{8}$AM and  $\red{9}$AM as follows:

$\qquad \dfrac{\textrm{rise}}{\textrm{run}} = \dfrac{\blue{20}-\blue{0}\ \textrm{ [km]}}{ \red{9} -\red{8} \ \textrm{ [h]}} = \dfrac{20 \textrm{ [km]}}{1 \textrm{ [h]}} = \purple{20} \ \textrm{km}/\textrm{h}$.

Sean maintains a speed of $\purple{20}\text{ km}$ per hour on his drive to work.

\paragraph{Hint 4}We can also calculate Sean's speed when he leaves work:

$\qquad \dfrac{\textrm{rise}}{\textrm{run}} = \dfrac{\blue{10}-\blue{0}\ \textrm{ [km]}}{ \red{19} -\red{17} \ \textrm{ [h]}} = \dfrac{10 \textrm{ [km]}}{2 \textrm{ [h]}} = \purple{5} \ \textrm{km}/\textrm{h}$.

Sean maintains a speed of $\purple{5}\text{ km/h}$ while stuck in traffic.

\paragraph{Hint 5}The correct way to complete the sentences is as follows.

Sean's work is located at a distance $\mathbf{20}\text{ km}$ from his home.   
Sean's speed when he was driving to work was $\mathbf{20}\text{ km/h}$.  
Sean's speed between $5$PM and $7$PM was 
$\mathbf{5}\text{ km/h}$**.**





\section{\href{https://www.khanacademy.org/devadmin/content/items/xd764f378}{xd764f378}}

The illustration below shows the graph of $y$ as a function of $x$.

\noindent $\langle$ img $\rangle$

**Complete the sentences based on the graph of the function.**

This is the graph of a [[☃ dropdown 3]] function.   
The $y$-intercept of the graph is $y=\:$[[☃ input-number 1]].  
The $x$-intercepts of the graph are at $x=\:$[[☃ input-number 2]] and $x=\:$[[☃ input-number 5]].  
The greatest value of $y$ is $y=\:$[[☃ input-number 3]] and it occurs when $x=\:$[[☃ input-number 4]].  
For $x$ between $x=2$ and $x=6$,  $y\:$[[☃ dropdown 1]]$\:0$.


Complete the sentences.

\paragraph{Hint 1}Let's use the graph to complete the sentences one by one.

\paragraph{Hint 2}A function which has a constant rate of change produces a graph that is a line and we say the function is *linear*. Otherwise, if the rate of change of the function is not constant, the graph will not be a line, and we say the function is *nonlinear*.

The graph shown is not a line. Therefore, we are looking at the graph of a **nonlinear** function.

\paragraph{Hint 3}The $y$-intercept of the graph corresponds to the value of the function when $\red{x=0}$.
The $y$-intercept of the graph is $\blue{y=-6}$.

\paragraph{Hint 4}The $x$-intercepts of the graph correspond to the values of $\red{x}$ for which $\blue{y=0}$. The $x$-intercepts of this graph are at $\red{x=2}$ and $\red{x=6}$.

\paragraph{Hint 5}The maximum value of $\blue{y}$ in the graph occurs at the top of the bump. The coordinates of the top of the bump are $(\red{4},\blue{2})$. Therefore, the maximum value is $\blue{y=2}$ and it occurs when $\red{x=4}$.

\paragraph{Hint 6}The graph of the function is above the $x$-axis
for values of $\red{x}$ between $\red{x=2}$ and $\red{x=6}$. For these values of $\red{x}$,  $y\geq 0$.

\paragraph{Hint 7}We can now complete the sentences.

This is the graph of a **nonlinear** function.   
The $y$-intercept of the graph is $y=\mathbf{-6}$.  
The $x$-intercepts of the graph are at $x=\mathbf{2}$ and $x=\mathbf{6}$.  
The greatest value of $y$ is $y=\mathbf{2}$ and it occurs when $x=\mathbf{4}$.  
For $x$ between $x=2$ and $x=6$,  $\mathbf{y\geq 0}$.







\section{\href{https://www.khanacademy.org/devadmin/content/items/xdba32ac1}{xdba32ac1}}

Korey is organizing a music concert. She plans to invest $\$2000$ to rent the venue and to pay the musicians and then sell the tickets for $\$20$ each. The graph below shows the profit $y$ she will make from the concert as a function of the number of tickets sold $x$.

\noindent $\langle$ img $\rangle$

If instead Korey decides to invest only $\$1200$ to organize the concert, **what will be the new graph of Korey's profit?**

[[☃ interactive-graph 1]]

Drag the two points to move the line into the correct position.

\paragraph{Hint 1}The first graph shows how the profit $\blue{y}$ varies as a function of the number of tickets sold $\red{x}$.
The slope of the graph is equal to $\purple{20}$, since Korey plans to sell the tickets for $\$\purple{20}$. The $y$-intercept of the graph is $\blue{y}=\green{-2000}$,
since she plans to invest $\$\green{2000}$ to organize the concert.

How will the graph of the profit change if Korey invests  only $\$\green{1200}$ for the concert organization?

\paragraph{Hint 2}Let's **find the equation** that describes Korey's profit when she invests $\$\green{2000}$ and sells the tickets for $\$\purple{20}$. 
In that case, the equation that describes the profit $\blue{y}$ as a function of the number of tickets sold $\red{x}$ is    

\begin{align*}
\qquad \blue{y} 
  &=\purple{m}\red{x}+\green{b} \\
  &=\purple{20}\cdot \red{x} \green{-2000}
\end{align*}  

The rate of change $\purple{m=20}$ corresponds to the price $\$\purple{20}$ per ticket, and the initial value $\green{b=-2000}$ corresponds to Korey's initial investment of $\$\green{2000}$.

\paragraph{Hint 3}If instead Korey invests $\$\green{1200}$ for the concert, then the equation of the profit $\blue{y}$ as a function of the number of tickets sold $\red{x}$ will become

\begin{align*}
\qquad \blue{y} 
  &=\purple{m}\red{x}+\green{b} \\
  &=\purple{20}\cdot \red{x} \green{-1200}
\end{align*}  

The initial value changed from $\green{b=-2000}$ to $\green{b=-1200}$ to reflect Korey's smaller up-front investment.

\paragraph{Hint 4}The graph of the new profit function $\blue{y}=\purple{20}\cdot \red{x} \green{-1200}$, passes through the point $(\red{0},\blue{-1200})$ and has slope equal to $\purple{20}$:

\noindent $\langle$ img $\rangle$

Note the slope of the graph remained the same and only the $y$-intercept of the graph changed.





\section{\href{https://www.khanacademy.org/devadmin/content/items/xe4b4df77}{xe4b4df77}}

The variable $y$ is proportional to the variable $x$. The constant of proportionality is $4$.

**Draw the graph of $y$ as a function of $x$.**  

[[☃ interactive-graph 1]]

Drag the two points to move the line into the correct position. 

\paragraph{Hint 1}Let's find the equation which corresponds to the verbal description and then draw the graph.

We are told the variable $\blue{y}$ is *proportional* to the variable $\red{x}$ which means we are looking for an equation of the form $\blue{y}=\purple{m}\cdot \red{x}$, where $\purple{m}$ is the constant of proportionality.

\paragraph{Hint 2}We are told the constant of proportionality is $\purple{m=4}$. Therefore, the equation which describes the relationship between $\blue{y}$ and $\red{x}$ is $\blue{y}=\purple{4}\cdot \red{x}$.

\paragraph{Hint 3}Note that the equation 
$\blue{y}=\purple{m}\cdot \red{x}$
is a special case of the linear equation $\blue{y}=\purple{m}\cdot \red{x} + \green{b}$, with the $y$-intercept $\green{b=0}$.

The graph of $\blue{y}$ versus $\red{x}$, for the equation $\blue{y}=\purple{4}\cdot\red{x}$ is a line which passes through the origin and has slope equal to $\purple{4}$:

\noindent $\langle$ img $\rangle$

\paragraph{Hint 4}





\section{\href{https://www.khanacademy.org/devadmin/content/items/xe931d345}{xe931d345}}

Jane sells hot dogs at football games. For last week's game, she bought hot dogs, buns and condiments for $\$6$ at the store, and sold the hot dogs for $\$1$ each at the game. The graph below shows the profit $y$ she made as a function of the number $x$ of hot dogs sold.

\noindent $\langle$ img $\rangle$

This week, Jane wants to change things up. She will buy bigger hot dogs, investing $\$9$ at the store, and sell the hot dogs for $\$2$ each. 

**What will be the graph of her profit for this week?**

[[☃ interactive-graph 1]]

Drag the two points to move the line into the correct position.

\paragraph{Hint 1}The first graph shows how the profit $\blue{y}$ varies as a function of the number of hot dogs sold $\red{x}$.
The slope of the graph is equal to $\purple{1}$, since Jane sold the hot dogs for $\$\purple{1}$ each last week. The $y$-intercept of the graph is $\green{y=-6}$,
since she invested $\$\green{6}$ for the food.

How will the graph of the profit change if Jane invests $\$\green{9}$ for the food and sells the hot dogs for $\$\purple{2}$ each?

\paragraph{Hint 2}First let's **find the equation** that describes the profit of last week's sales. The equation that describes the profit $\blue{y}$ as a function of the number of hot dogs sold $\red{x}$ is    

\begin{align*}
\qquad \blue{y} 
  &=\purple{m}\red{x}+\green{b} \\
  &=\purple{1}\cdot \red{x} \green{-6}
\end{align*}  

The slope $\purple{m=1}$ corresponds to the price $\$\purple{1}$ per hot dog, and the initial value $\green{b=-6}$ corresponds to Jane's initial investment of $\$\green{6}$.

\paragraph{Hint 3}If this week Jane invests $\$\green{9}$ for the food
and sells the hot dogs for $\$\purple{2}$ each,
then the equation of the profit $\blue{y}$ as a function of the number of hot dogs sold $\red{x}$ will become

\begin{align*}
\qquad \blue{y} 
  &=\purple{m}\red{x}+\green{b} \\
  &=\purple{2}\cdot \red{x} \green{-9}
\end{align*}  

The initial value changed from $\green{b=-6}$ to $\green{b=-9}$ to reflect the fact that Jane spent  $\$\green{9}$ at the store this week.
The slope changed to $\purple{m=2}$ since Jane sells the hot dogs for $\$\purple{2}$ each this week.


\paragraph{Hint 4}The graph of the new profit function $\blue{y}=\purple{2}\cdot \red{x} \green{-9}$, passes through the point $(\red{0},\blue{-9})$ and has slope equal to $\purple{2}$:

\noindent $\langle$ img $\rangle$






\section{\href{https://www.khanacademy.org/devadmin/content/items/xf16c46c4}{xf16c46c4}}

A doctor observes a graph that shows the electrical activity (in Volts) of the heart of a patient over a period of time. Each spike corresponds to one heart beat.

\noindent $\langle$ img $\rangle$

The doctor needs to calculate the heart rate of the patient in *beats per minute*.   
**What is the heart rate of this patient?**

[[☃ input-number 1]] beats per minute

\paragraph{Hint 1}The graph shown is called an *electrocardiogram*:
a diagram which shows the electric voltage of the heart beat. Each spike in the graph represents one heart beat. 

To find the heart rate in *beats per minute*, we need to calculate how many heart beats will occur in $\purple{60}$ seconds.

\paragraph{Hint 2}Looking at the graph, we observe that $\blue{6}$ heart beats occur in each period of $\red{6}$ seconds.
To find the number of heart beats in one minute we proceed as follows:

$\displaystyle \dfrac{\blue{6} \ \textrm{beats} }{\red{6} \ \cancel{\textrm{sec}}} \cdot \dfrac{ \purple{60} \ \cancel{\textrm{sec}}}{1 \ \textrm{min}} = \dfrac{ \cancel{6} \cdot 60 \ \textrm{beats}}{ \cancel{6} \ \textrm{min}} = 60$ beats per minute.

This makes sense: $60$ beats per minute is equivalent to $1$ beat every second, which is what we see in the graph.

\paragraph{Hint 3}The heart rate of the patient is $60$ beats per minute.






\section{\href{https://www.khanacademy.org/devadmin/content/items/xf17c366e}{xf17c366e}}

It takes two slices of bread to make a sandwich. 
When you are making sandwiches, the number of slices of bread you will need is a function of the number of sandwiches you want to prepare. Assume $x$ represents the number of sandwiches you want to make, and $y$ represents the number of slices of bread.

**Draw the graph that represents this function.**

[[☃ interactive-graph 1]]

Drag the two points to move the line into the correct position. 

\paragraph{Hint 1}We’re looking for the graph of the function which describes how the number of slices of bread $\blue{y}$ depends on the number of sandwiches you want to make $\red{x}$.

\paragraph{Hint 2}The number of slices you will need $\blue{y}$ is *proportional* to the number of sandwiches you want to make $\red{x}$. We can think of $\blue{y}$ as a function of $\red{x}$. The equation of this function is

 \begin{align*}
\quad \blue{y} 
 &= \purple{m}\cdot\red{x} \\[1mm]
 &= \purple{2}\cdot\red{x} 
\end{align*}

The constant of proportionality is $\purple{m=2}$ since it takes two slices of bread to make one sandwich.

\paragraph{Hint 3}The graph of the function $ \blue{y} = \purple{2}\red{x}$ is a line which passes through the origin and has slope equal to $\purple{2}$:   
\noindent $\langle$ img $\rangle$

\paragraph{Hint 4}





\section{\href{https://www.khanacademy.org/devadmin/content/items/xf4f6c84c}{xf4f6c84c}}

A doctor observes a graph that shows the electrical activity (in Volts) of the heart of a patient over a period of time. Each spike corresponds to one heart beat.

\noindent $\langle$ img $\rangle$

The doctor needs to calculate the heart rate of the patient in *beats per minute*.   
**What is the heart rate of this patient?**

[[☃ input-number 1]] beats per minute

\paragraph{Hint 1}The graph shown is called an *electrocardiogram*:
a diagram which shows the electric voltage of the heart beat. Each spike in the graph represents one heart beat. 

To find the heart rate in *beats per minute*, we need to calculate how many heart beats will occur in $\purple{60}$ seconds.

\paragraph{Hint 2}Looking at the graph, we observe that $\blue{9}$ heart beats occur in each period of $\red{6}$ seconds.
To find the number of heart beats in one minute we proceed as follows:

$\displaystyle \dfrac{\blue{9} \ \textrm{beats} }{\red{6} \ \cancel{\textrm{sec}}} \cdot \dfrac{ \purple{60} \ \cancel{\textrm{sec}}}{1 \ \textrm{min}}
 = \dfrac{ 9  \cdot 60 \ \textrm{beats}}{ 6 \ \textrm{min}} 
=\dfrac{ 9  \cdot \cancel{6} \cdot 10 \ \textrm{beats}}{ \cancel{6} \ \textrm{min}} = 90$ beats per minute.


\paragraph{Hint 3}The heart rate of the patient is $90$ beats per minute.






\section{\href{https://www.khanacademy.org/devadmin/content/items/xf8bb6f6a}{xf8bb6f6a}}

The variable $y$ is related to the variable $x$ through a linear equation. The graph of $y$ as a function of $x$ passes through the points $(1,3)$ and $(2,5)$.

**Draw the graph of this function.**  

[[☃ interactive-graph 1]]

Drag the two points to move the line into the correct position. 

\paragraph{Hint 1}To draw the graph of $\blue{y}$ versus $\red{x}$, drag one of the points of the graph placing it at the point $(\red{1},\blue{3})$ and drag the second point to the coordinates $(\red{2},\blue{5})$.

\noindent $\langle$ img $\rangle$

Wow, that was easy!

\paragraph{Hint 2}The graph of this line corresponds to some linear equation: $\blue{y}=\purple{m}\cdot \red{x} + \green{b}$, where $\purple{m}$ is the slope and $\green{b}$ is the $y$-intercept.

Looking at the graph of the line that passes through the points $(\red{1},\blue{3})$ and $(\red{2},\blue{5})$,
we can easily read off the value of the slope $\purple{m}$ and the $y$-intercept $\green{b}$. The equation of the line in the graph is $\blue{y}=\purple{2}\cdot \red{x} + \green{1}$.



\end{document}

