% Mon 19 Aug 2013 14:33:12 EDT
%DIF LATEXDIFF DIFFERENCE FILE
%DIF DEL writing_expresions_3.tex         Mon Aug 19 14:41:54 2013
%DIF ADD writing_expresions_3_after.tex   Mon Aug 19 15:02:14 2013

\documentclass[twocolumn,10pt]{article}
\title{Writing expressions 3}
\setlength{\columnsep}{20pt} 
\usepackage{amsmath,hyperref,cancel,graphicx}
 \def\shrinkfactor{0.55}
 \usepackage[margin=1.5cm]{geometry}
\usepackage[usenames,dvipsnames]{color}
 
 \newcommand{\blue}[1]{{\color{Blue}#1}} 
 \newcommand{\purple}[1]{{\color{Purple}#1}} 
 \newcommand{\red}[1]{{\color{Red}#1}} 
 \newcommand{\green}[1]{{\color{Green}#1}} 
 \newcommand{\gray}[1]{{\color{Gray}#1}} 
  \newcommand{\pink}[1]{{\color{Magenta}#1}}   
%DIF PREAMBLE EXTENSION ADDED BY LATEXDIFF
%DIF UNDERLINE PREAMBLE %DIF PREAMBLE
\RequirePackage[normalem]{ulem} %DIF PREAMBLE
\RequirePackage{color}\definecolor{RED}{rgb}{1,0,0}\definecolor{BLUE}{rgb}{0,0,1} %DIF PREAMBLE
\providecommand{\DIFadd}[1]{{\protect\color{blue}\uwave{#1}}} %DIF PREAMBLE
\providecommand{\DIFdel}[1]{{\protect\color{red}\sout{#1}}}                      %DIF PREAMBLE
%DIF SAFE PREAMBLE %DIF PREAMBLE
\providecommand{\DIFaddbegin}{} %DIF PREAMBLE
\providecommand{\DIFaddend}{} %DIF PREAMBLE
\providecommand{\DIFdelbegin}{} %DIF PREAMBLE
\providecommand{\DIFdelend}{} %DIF PREAMBLE
%DIF FLOATSAFE PREAMBLE %DIF PREAMBLE
\providecommand{\DIFaddFL}[1]{\DIFadd{#1}} %DIF PREAMBLE
\providecommand{\DIFdelFL}[1]{\DIFdel{#1}} %DIF PREAMBLE
\providecommand{\DIFaddbeginFL}{} %DIF PREAMBLE
\providecommand{\DIFaddendFL}{} %DIF PREAMBLE
\providecommand{\DIFdelbeginFL}{} %DIF PREAMBLE
\providecommand{\DIFdelendFL}{} %DIF PREAMBLE
%DIF END PREAMBLE EXTENSION ADDED BY LATEXDIFF

\begin{document}
\maketitle



\section{\href{https://www.khanacademy.org/devadmin/content/items/x0e26cc9b}{x0e26cc9b}}

\noindent
The volume of a cube is equal to its length times its width times its height. A cube with side $x$ has length $\red{x}$, width $\green{x}$, and height $\blue{x}$, so its volume is equal to $\red{x}\cdot \green{x} \cdot \blue{x} =x^3$. 

Shane takes a big cube of butter with side length $5$ from the fridge. He then cuts out a little cube of side length $2$ from one of the corners of the cube.

**Write an expression that describes the volume of butter left**. 

\paragraph{Ans}  $5^3 - 2^3$ 

\paragraph{Hint 1}To find the volume of butter left, we need to find the volume of the big cube of butter and subtract the volume of the little cube that was cut out.

\paragraph{Hint 2}The volume of a cube of side $x$ is equal to $\red{x}\cdot \green{x} \cdot \blue{x} =\purple{x^3}$, so the volume of the big cube of butter is $\purple{5^3}$.

\paragraph{Hint 3}The volume of the small cube that Shane cut out is $\pink{2^3}$. 

\paragraph{Hint 4}The volume of butter remaining is described by the expression:   $\purple{5^3}-\pink{2^3}$, since $\purple{5^3}$ is the volume of butter Shane started with, and $\pink{2^3}$ is the volume of butter he cut out.



\medskip
\noindent
\textbf{Tags:} {\footnotesize CC.6.EE.A.1, CC.6.EE.A.2c, SB.6.1.E.2.CR, Writing exponential expressions 1, Expressions in word problems}\\
\textbf{Version:} 818a4425.. 2013-06-22
\smallskip\hrule





\section{\href{https://www.khanacademy.org/devadmin/content/items/x105b8ba7}{x105b8ba7}}

\noindent
Write an expression which corresponds to  ** *x* squared plus *z* cubed.**

\paragraph{Ans}  $x^2 + z^3$ 

\paragraph{Hint 1}We are given a verbal expression with two terms. 
Let's take the terms one by one.

\paragraph{Hint 2}The first term is ** *x* squared**, which is the number $\red{x}$ multiplied with itself: $\red{x} \cdot \red{x}  = \red{x}^{\blue{2}}$. 

\paragraph{Hint 3}The second term is ** *z* cubed**, which is the same as $\pink{z} \cdot \pink{z} \cdot \pink{z} = \pink{z}^\purple{3}$. This is the number $\pink{z}$ raised to the power $\purple{3}$, like in the formula for the volume of a cube.

\paragraph{Hint 4}So we can write **$\red{x}$ $\blue{\textrm{squared}}$ plus $\pink{z}$ $\purple{\textrm{cubed}}$** as 
$\red{x}^\blue{2} + \pink{z}^\purple{3}$.



\medskip
\noindent
\textbf{Tags:} {\footnotesize SB.6.1.E.2.CR, Writing exponential expressions 1, CC.6.EE.A.2a, Transcribing expressions, CC.6.EE.A.2}\\
\textbf{Version:} 65a1e011.. 2013-06-22
\smallskip\hrule





\section{\href{https://www.khanacademy.org/devadmin/content/items/x321fe412}{x321fe412}}

\noindent
The area of a square is equal to its width times its height. A square with side, $x$, has width $\green{x}$ and height $\blue{x}$, so its area is equal to $\green{x} \cdot \blue{x} =x^2$. 

Xiao Yi is designing a new logo for her company. She starts from  a big square with side length $a$ and cuts out a smaller square of side length $b$ in the center to make a giant letter O.

**Write an expression that describes the area of the logo.** 

\paragraph{Ans}  $a^2 - b^2$ 

\paragraph{Hint 1}To find the area of the logo, let's first find the area of the big square and then subtract the area of the square that Xiao Yi cut out.

\paragraph{Hint 2}We know the area of a square is equal to its side length squared,
so the area of the square with side length $\blue{a}$ is $\blue{a^2}$. 

\paragraph{Hint 3}The area of the little square with side length $\green{b}$, which she cut out from the center, is $\green{b^2}$. 

\paragraph{Hint 4}We can now subtract these two numbers to find the area remaining: $\blue{a^2}-\green{b^2}$. This is the area of the big square that remains after Xiao Yi cut out the piece in the center.




\medskip
\noindent
\textbf{Tags:} {\footnotesize CC.6.EE.A.1, CC.6.EE.A.2c, SB.6.1.E.2.CR, Writing exponential expressions 1, Expressions in word problems}\\
\textbf{Version:} 908cc517.. 2013-06-22
\smallskip\hrule





\section{\href{https://www.khanacademy.org/devadmin/content/items/x4f44471b}{x4f44471b}}

\noindent
Write a mathematical expression which represents **four to the $9^{\textrm{th}}$ power minus nine to the $4^{\textrm{th}}$ power.**


\paragraph{Ans}  $4^9 - 9^4$ 

\paragraph{Hint 1}We are given a verbal expression with two terms involving numbers with exponents. Let's take the terms one by one.

\paragraph{Hint 2}The  $\blue{9^{\textrm{th}}}$ $\blue{\textrm{power}}$ of a number is the number multiplied with itself $\blue{9}$ times,
so **$\green{\textrm{four}}$ to the $\blue{9^{\textrm{th}}}$ power**
is the same as $\green{4} \cdot \green{4} \cdot \green{4} \cdot \green{4} \cdot \green{4} \cdot \green{4} \cdot \green{4} \cdot \green{4} \cdot \green{4}$, which is written as $\green{4}^{\blue{9}}$ for short. 

\paragraph{Hint 3}The second term consists of the number $\red{\textrm{nine}}$ raised to the $\purple{4^{\textrm{th}}}$ $\purple{\textrm{power}}$, which is written as $\red{9}^{\purple{4}}$.


\paragraph{Hint 4}The final answer is equal to the first term **minus** the second term. The expression is written as $\green{4}^{\blue{9}} - \red{9}^{\purple{4}}$.



\medskip
\noindent
\textbf{Tags:} {\footnotesize CC.6.EE.A.1, SB.6.1.E.2.CR, Writing exponential expressions 1, CC.6.EE.A.2a, Transcribing expressions}\\
\textbf{Version:} 9022cd9d.. 2013-06-22
\smallskip\hrule





\section{\href{https://www.khanacademy.org/devadmin/content/items/x5760f8f2}{x5760f8f2}}

\noindent
An annual visit to the dentist could be very expensive. The basic price for the annual checkup and cleaning is just $\$50$, but if the dentist finds any cavities, there will be an additional charge of $\$100$ for each cavity your dentist will have to fill.

If the dentist finds $n$ cavities, **what will be the cost of the visit?**

\paragraph{Ans}  $50+100*n$ 

\paragraph{Hint 1}Don't worry, you are not at the dentist!   
This is just a math exercise. 

Let's see how to calculate the total cost of the visit. The price for fixing each cavity is $\$\red{100}$, so the cost of fixing  $\purple{n}$ cavities is going to be $\red{100}\purple{n}$ dollars.

\paragraph{Hint 2}We have to add to this the basic cost of the checkup and  cleaning which is $\$\green{50}$.

\paragraph{Hint 3}The total cost of the visit, in dollars, is described by the math expression $\red{100}\purple{n}+\green{50}$.

\paragraph{Hint 4}A good strategy to avoid spending time and money at the dentist is to eat less sweet stuff so that $\purple{n}$ is a small number. Bonus points if you can hit $\purple{n=0}$!



\medskip
\noindent
\textbf{Tags:} {\footnotesize SB.6.1.E.2.CR, Writing expressions 1, Expressions in word problems, CC.6.EE.A.2}\\
\textbf{Version:} 75b01415.. 2013-06-21
\smallskip\hrule





\section{\href{https://www.khanacademy.org/devadmin/content/items/x5d81ab19}{x5d81ab19}}

\noindent
Alex likes to eat sushi. At his favorite sushi restaurant, the price customers pay includes a base charge of $\$5$ for tea and reservation plus $\$2$ per sushi roll ordered. The following table describes the price list:

number of sushi rolls | price
:-: | :-: 
$0$ | $5=5+2\cdot 0$
$1$ | $7=5+2\cdot 1$
$2$ | $9=5+2\cdot 2$
$3$ | $11=5+2\cdot 3$
$4$ | $13=5+2\cdot 4$
$5$ | $15=5+2\cdot 5$
$6$ | $17=5+2\cdot 6$

**What is the expression that describes the price of eating $n$ sushi rolls at this restaurant?**

\paragraph{Ans}  $5+2*n$ 

\paragraph{Hint 1}Let�s look at the price Alex paid at the restaurant and how it varies with the number of sushi rolls he orders.
We know there is a base charge of $\red{5}$ dollars and an additional $\purple{2}$ dollars per roll.

\paragraph{Hint 2}When Alex orders one sushi roll ($\blue{n=1}$), the price is $7$ dollars.
This price is obtained by combining the base charge of $\red{5}$ plus the price for one sushi roll: $\red{5}+\purple{2}\cdot \blue{1} = 7$.

\paragraph{Hint 3}If Alex orders $\blue{n=2}$ sushi rolls, the price goes up to $9$ dollars, since $\red{5}+\purple{2}\cdot \blue{2}=5+4 = 9$.

\paragraph{Hint 4}The price calculation works the same way for **any** number of rolls. We start with the base price of $5$ dollars and add two times the number of sushi rolls.

\paragraph{Hint 5}The expression which corresponds to the price of $\blue{n}$ sushi rolls at this restaurant is $\red{5}+\purple{2}\blue{n}$. 

Note how the mathematical expression $\red{5}+\purple{2}\blue{n}$ is much easier to remember than trying to remember the whole price list.



\medskip
\noindent
\textbf{Tags:} {\footnotesize SB.6.1.E.2.CR, Writing expressions 1, Abstracting expression from a table, CC.6.EE.A.2}\\
\textbf{Version:} 08f1e6ac.. 2013-06-19
\smallskip\hrule





\section{\href{https://www.khanacademy.org/devadmin/content/items/x65b8321b}{x65b8321b}}

\noindent
Enter a mathematical expression which corresponds to ** *x* cubed plus *z* to the $7^{\textrm{th}}$ power.**

\paragraph{Ans}  $x^3 + z^7$ 

\paragraph{Hint 1}Positive exponents (powers) tell us how many times to multiply a number by itself. We are given a verbal expression with two terms with exponents. Let's take the terms one by one.

\paragraph{Hint 2}The first term is ** *x* cubed,** which is the same as the number $\red{x}$ multiplied together $\blue{3}$ times 
$\red{x} \cdot \red{x} \cdot \red{x} = \red{x}^{\blue{3}}$.

Cubed means **raised to the  $\blue{3^{\textrm{rd}}}$ power**  like in the formula for the volume of a cube.


\paragraph{Hint 3}The second term is ** *z* to the $\purple{7^{\textrm{th}}}$ power**, which is the same as the number $\pink{z}$ multiplied by itself $\purple{7}$ times: $\pink{z} \cdot \pink{z} \cdot \pink{z} \cdot \pink{z} \cdot \pink{z} \cdot \pink{z} \cdot \pink{z} = \pink{z}^\purple{7}$.

\paragraph{Hint 4}So we can write **$\red{x}$ $\blue{\textrm{cubed}}$ plus $\pink{z}$ to the $\purple{7^{\textrm{th}}}$ power** as 
$\red{x}^\blue{3} + \pink{z}^\purple{7}$.



\medskip
\noindent
\textbf{Tags:} {\footnotesize SB.6.1.E.2.CR, Writing exponential expressions 1, CC.6.EE.A.2a, Transcribing expressions, CC.6.EE.A.2}\\
\textbf{Version:} 163f36b7.. 2013-06-22
\smallskip\hrule





\section{\href{https://www.khanacademy.org/devadmin/content/items/x6d1f513b}{x6d1f513b}}

\noindent
Marie-Jeanne is the system administrator of a big website.
The website requires multiple web servers and database servers to run. Each web server costs $\$40$ per month and each database server costs $\$60$ per month.

To get an idea of the monthly bill her company will have to pay,  Marie-Jeanne makes a little table: 

web servers | database servers | monthly hosting bill
:-: | :-: | :-: 
$1$ | $1$ | $40\cdot1+60\cdot1=100$ 
$2$ | $1$ | $40\cdot2+60\cdot1 =140$ 
$3$ | $1$ | $40\cdot3+60\cdot1 =180$  
$3$ | $2$ | $40\cdot3+60\cdot2 =240$  
$5$ | $2$ | $40\cdot5+60\cdot2 =320$  
$10$ | $5$ | $40\cdot10+60\cdot5 =700$  
$20$ | $10$ | $40\cdot20+60\cdot10 =1400$  

**What is the expression that describes the monthly hosting bill for $n$ web servers and $m$ database servers?**

\paragraph{Ans} The expression for the monthly hosting bill (in dollars) is
[[? expression 1]]   40*n + 60*m

\paragraph{Hint 1}The expression of the monthly hosting bill depends on how many web servers and how many database servers are used. Each web server costs $\red{40}$ dollars per month and each database server costs $\purple{60}$ dollars per month.

\paragraph{Hint 2}If we define $\blue{n}$ to be the number of web servers, and $\green{m}$ to be the number of database servers, the monthly hosting bill will be $\red{40}\cdot \blue{n}+\purple{60}\cdot \green{m}$ dollars.

\paragraph{Hint 3}The math expression $\red{40}\cdot \blue{n}+\purple{60}\cdot \green{m}$ will allow Marie-Jeanne to calculate the monthly price for *any* combination of servers.



\medskip
\noindent
\textbf{Tags:} {\footnotesize SB.6.1.E.2.CR, Writing expressions 1, Abstracting expression from a table, CC.6.EE.A.2}\\
\textbf{Version:} 8e0b6b00.. 2013-06-22
\smallskip\hrule





\section{\href{https://www.khanacademy.org/devadmin/content/items/x759767f4}{x759767f4}}

\noindent
Write a mathematical expression which corresponds to ** *a* times *b*, minus four times *c*, plus $20$ divided by *d***.

\paragraph{Ans}  $a*b-4c+20/d$ 

\paragraph{Hint 1}We are given the verbal description of an expression with three terms. Let's take the terms one by one, and write them out in math.

\paragraph{Hint 2}The first term is ** *a* times *b* ** which is the product of the two variables $\red{a}$ and $\blue{b}$, which is written as $\red{a}\cdot\blue{b}$ or without the dot: $\red{a}\blue{b}$. 

\paragraph{Hint 3}Next we have to **subtract** the second term ** four times *c* **, which is written as $-\purple{4}\green{c}$ in math.

\paragraph{Hint 4}Finally, we have to **add** the term **$20$ divided by *d***, which we will write as $\mathbf{+}\dfrac{\pink{20}}{\gray{d}} $.

\paragraph{Hint 5}We can therefore write the whole expression 
** *a* times *b*, minus four times *c*, plus $20$ divided by *d* ** as $\red{a}\blue{b} - \purple{4}\green{c} + \dfrac{\pink{20}}{\gray{d}}$.



\medskip
\noindent
\textbf{Tags:} {\footnotesize SB.6.1.E.2.CR, Writing expressions 1, CC.6.EE.A.2a, Transcribing expressions, CC.6.EE.A.2}\\
\textbf{Version:} 2985afaa.. 2013-06-22
\smallskip\hrule





\section{\href{https://www.khanacademy.org/devadmin/content/items/x856db40d}{x856db40d}}

\noindent
The volume of a cube is equal to its length times its width times its height. A cube with side $x$ has length $\red{x}$, width $\green{x}$, and height $\blue{x}$, so its volume is equal to $\red{x}\cdot \green{x} \cdot \blue{x} =x^3$. 

You have two cubes that you fill with water to make ice cubes. The first cube has a side length $6$. The second cube has a side length $5$. 

**Write an expression which represents the total volume of ice you can make.**

\paragraph{Ans}  $6^3 + 5^3$ 

\paragraph{Hint 1}To find the total volume of ice, let's write the expression for the volume of each cube and then add the two expressions together.

\paragraph{Hint 2}We know the volume of a cube is equal to the length of its side raised to the third power. 

The volume of the first cube, with side length $\blue{6}$, is $\blue{6^3}$. 

\paragraph{Hint 3}The volume of the second cube, with side length $\green{5}$, is $\green{5^3}$. 

\paragraph{Hint 4}We can now add the volumes of the two cubes to get the total volume of ice cubes: $\blue{6^3}+\green{5^3}$. 




\medskip
\noindent
\textbf{Tags:} {\footnotesize CC.6.EE.A.1, CC.6.EE.A.2c, SB.6.1.E.2.CR, Writing exponential expressions 1, Expressions in word problems}\\
\textbf{Version:} 4b7e5491.. 2013-06-22
\smallskip\hrule





\section{\href{https://www.khanacademy.org/devadmin/content/items/x8ad8b817}{x8ad8b817}}

\noindent
The volume of a cube is equal to its length times its width times its height. A cube with side $x$ has length $\red{x}$, width $\green{x}$, and height $\blue{x}$, so its volume is equal to $\red{x} \cdot \green{x} \cdot \blue{x} =x^3$. 

You have two cubes that you fill with water to make ice cubes. The first cube has a side length $a$. The second cube has a side length $b$. 
**Write an expression which represents the total volume of ice you can make.**

\paragraph{Ans}  $a^3 + b^3$ 

\paragraph{Hint 1}To find the total volume of ice, let's write the expression for the volume of each cube and then add the two expressions together.

\paragraph{Hint 2}We know the volume of a cube is equal to the length of its side raised to the third power. 

The volume of the first cube, with side length $\purple{a}$, is $\purple{a^3}$. 

\paragraph{Hint 3}The volume of the second cube, with side length $\pink{b}$, is $\pink{b^3}$. 

\paragraph{Hint 4}We can now add the volumes of the two cubes to get the total volume of ice cubes: $\purple{a^3}+\pink{b^3}$. 




\medskip
\noindent
\textbf{Tags:} {\footnotesize SB.6.1.E.2.CR, Writing exponential expressions 1, Expressions in word problems, CC.6.EE.A.2}\\
\textbf{Version:} 2142760b.. 2013-06-22
\smallskip\hrule





\section{\href{https://www.khanacademy.org/devadmin/content/items/x8b2fd182}{x8b2fd182}}

\noindent
Write a mathematical expression which corresponds to ** *x* times *y*, minus the product of *a* times *b* times *c* **.

\paragraph{Ans}  $x*y-a*b*c$ 

\paragraph{Hint 1}We are given a verbal expression which is the **difference** of two terms.  Let's take the terms one by one, and write them out in math.

\paragraph{Hint 2}The first term is ** *x* times *y* ** which is the product of the two variables $\red{x}$ and $\blue{y}$, which is written as $\red{x}\cdot\blue{y}$ or without the dot: $\red{x}\blue{y}$. 

\paragraph{Hint 3}The second term ** *a* times *b* times *c* ** is the product of the three variables $\purple{a}$, $\green{b}$, and $\pink{c}$, which is written as $\purple{a}\green{b}\pink{c}$ in math.

\paragraph{Hint 4}We can therefore write ** *x* times *y*, $\gray{\textrm{minus}}$ the product of *a* times *b* times *c* ** as $\red{x}\blue{y} \gray{-} \purple{a}\green{b}\pink{c}$.



\medskip
\noindent
\textbf{Tags:} {\footnotesize SB.6.1.E.2.CR, Writing expressions 1, CC.6.EE.A.2a, Transcribing expressions, CC.6.EE.A.2}\\
\textbf{Version:} 145f3424.. 2013-07-16
\smallskip\hrule





\section{\href{https://www.khanacademy.org/devadmin/content/items/x9239a4fe}{x9239a4fe}}

\noindent
Write a mathematical expression which corresponds to **five-thirds plus one-quarter**.

\paragraph{Ans}  $5/3+1/4$ 

\paragraph{Hint 1}We are given a verbal expression which is the **sum** of two fractions. Let's take the terms one by one and write out the expression in math.

\paragraph{Hint 2}The first term $\red{\textrm{five}}$-$\blue{\textrm{thirds}}$ is written as the fraction $\dfrac{ \red{5} }{ \blue{3} }\,$.

\paragraph{Hint 3}The second term $\pink{\textrm{one}}$-$\purple{\textrm{quarter}}$ is written as $\dfrac{ \pink{1} }{ \purple{4}}\,$. 

The expression we are looking for is the **sum** of these two terms.

\paragraph{Hint 4}We write **five-thirds plus one-quarter** as
$\dfrac{ \red{5} }{ \blue{3} } + \dfrac{ \pink{1} }{ \purple{4}}\,$. 



\medskip
\noindent
\textbf{Tags:} {\footnotesize CC.6.EE.A.1, SB.6.1.E.2.CR, Writing expressions 1, CC.6.EE.A.2a, Transcribing expressions}\\
\textbf{Version:} 080b29da.. 2013-06-22
\smallskip\hrule





\section{\href{https://www.khanacademy.org/devadmin/content/items/x9b1b8b09}{x9b1b8b09}}

\noindent
A company has \DIFdelbegin \DIFdel{fixed monthly **revenues** }\DIFdelend \DIFaddbegin \DIFadd{monthly revenues }\DIFaddend of $1000$ dollars and \DIFdelbegin \DIFdel{variable monthly **costs**. }%DIFDELCMD < 

%DIFDELCMD < %%%
\DIFdelend \DIFaddbegin \DIFadd{monthly costs described by the variable $c$. }\DIFaddend The company�s profits are equal to the revenues minus the costs. 
\DIFdelbegin \DIFdel{If in a given month the company had costs of $c$ dollars, **write }\DIFdelend \DIFaddbegin 

\DIFadd{**Write }\DIFaddend an expression which describes the company�s monthly profits.** 

\paragraph{Ans}  $1000 - c$ 

\paragraph{Hint 1}We know the company has monthly revenues of $\blue{1000}$ dollars. To find the profits, we have to subtract the costs from the revenue.

\paragraph{Hint 2}Since the monthly costs of the company are represented by the variable $\red{c}$, the revenues minus costs gives $\blue{1000} - \red{c}$.

\paragraph{Hint 3}So, the company's monthly profits are described by the expression $\blue{1000} - \red{c}$. 



\medskip
\noindent
\textbf{Tags:} {\footnotesize SB.6.1.E.2.CR, Writing expressions 1, Expressions in word problems, CC.6.EE.A.2}\\
\textbf{Version:} \DIFdelbegin \DIFdel{889088bd.. 2013-06-22
}\DIFdelend \DIFaddbegin \DIFadd{c7769d36.. 2013-08-19
}\DIFaddend \smallskip\hrule





\section{\href{https://www.khanacademy.org/devadmin/content/items/x9f2aedaa}{x9f2aedaa}}

\noindent
Write an expression to represent **four squared plus two cubed.**


\paragraph{Ans}  $4^2 + 2^3$ 

\paragraph{Hint 1}We are given a verbal expression with two terms with exponents. Let's take the terms one by one.

\paragraph{Hint 2}**Four squared** is the same as $4 \cdot 4$, which we can write with exponents as $\blue{4^2}$.

\paragraph{Hint 3}The second term **two cubed**, is the same as $2 \cdot 2 \cdot 2$, which we can write with exponents as $\red{2^3}$.

\paragraph{Hint 4}So we can write **four squared plus two cubed** as $\blue{4^2} + \red{2^3}$.



\medskip
\noindent
\textbf{Tags:} {\footnotesize CC.6.EE.A.1, SB.6.1.E.2.CR, Writing exponential expressions 1, CC.6.EE.A.2a, Transcribing expressions}\\
\textbf{Version:} 2c7e5358.. 2013-06-22
\smallskip\hrule





\section{\href{https://www.khanacademy.org/devadmin/content/items/xa1b0642b}{xa1b0642b}}

\noindent
Samurai swords are made by repeatedly heating a steel bar, hammering it, and then folding it in half. This process creates a sword with many layers.
Each time you fold the steel, the number of layers doubles. Folding a steel bar  $n$ times produces 
a sword with $2^n$ layers.

**Write an expression that corresponds to the number of layers in a piece of steel that has been folded $15$ times.**

\paragraph{Ans}  $2^15$ 

\paragraph{Hint 1}We want to find the total number of layers of steel in the sword.

\paragraph{Hint 2}The formula tells us that folding the steel $\blue{n}$ times produces $2^{\blue{n}}$ layers. So if you fold the steel $\green{\textrm{once}}$, there will be $2^{\green{1}}=2$ layers in it. If you fold it $\pink{\textrm{twice}}$, there will be $2^{\pink{2}}=4$ layers. If you fold it $\purple{3}$ times, there will be $2^{\purple{3}}=8$ layers, etc.

\paragraph{Hint 3}The sword is made from a piece of steel which was folded $\red{15}$ times so there are $2^{\red{15}}$ layers of steel in it.



\medskip
\noindent
\textbf{Tags:} {\footnotesize CC.6.EE.A.1, CC.6.EE.A.2c, SB.6.1.E.2.CR, Writing exponential expressions 1, Expressions in word problems, CC.6.EE.A.2}\\
\textbf{Version:} 1666f4bb.. 2013-07-16
\smallskip\hrule





\section{\href{https://www.khanacademy.org/devadmin/content/items/xa2083c29}{xa2083c29}}

\noindent
Tommy wants to take a bath. The bathtub is filling with water at the rate of $60$ liters per minute.
**Write an expression which describes how many liters of water are in the bathtub after $t$ minutes**. 

\paragraph{Ans}  $60*t$ 

\paragraph{Hint 1}The bathtub is filling up at the rate of $\blue{60}$ liters per minute. The time $\red{t}$ is measured in minutes.

\paragraph{Hint 2}To find how much water is in the bathtub at time $\red{t}$, we need to multiply the rate at which we are filling the bathtub by the time $\red{t}$.

\paragraph{Hint 3}The expression which corresponds to the total volume of water in the bathtub is $\blue{60}\cdot\red{t}$.

\paragraph{Hint 4}Tommy can easily check that the units for the expression $\blue{60}\cdot \red{t}$ are indeed liters.

The number $\blue{60}$ is measured in 
$\dfrac{ \purple{\textrm{liters}} }{ \pink{\textrm{minute}} }$, and the time $\red{t}$ is measured in $\pink{\textrm{minute}}$s so the units of  $\pink{\textrm{ minute}}$s cancel out when we compute the product:

$\qquad \displaystyle 
\blue{60} 
\frac{ \purple{\textrm{liters}} }{ \cancel{\pink{\textrm{ minute}}} } 
\cdot \red{t}\  \cancel{\pink{\textrm{ minute}}}  
= \blue{60}\cdot\red{t} \ \purple{\textrm{ liters}}$. 



\paragraph{Hint 5}So the volume of water in the bathtub after $\red{t}$ minutes is $\blue{60}\cdot\red{t}$ liters.



\medskip
\noindent
\textbf{Tags:} {\footnotesize SB.6.1.E.2.CR, Writing expressions 1, Expressions in word problems, CC.6.EE.A.2}\\
\textbf{Version:} ebef1d4f.. 2013-07-16
\smallskip\hrule





\section{\href{https://www.khanacademy.org/devadmin/content/items/xa4194dcd}{xa4194dcd}}

\noindent
Write a mathematical expression which corresponds to **fifty-five raised to the fifth power minus sixteen times six**.

\paragraph{Ans}  $55^5-16*6$ 

\paragraph{Hint 1}We are given a verbal expression which is the **difference** of two terms.   
 Let's take the terms one by one and write them out in math.

\paragraph{Hint 2}The first term **$\red{\textrm{fifty-five}}$ raised to the $\blue{\textrm{fifth}}$ power** is written $\red{55}^{\blue{5}}$.
Positive powers of numbers tell us how many times to multiply the number by itself: $\red{55}^{\blue{5}}=\red{55}\cdot\red{55}\cdot\red{55}\cdot\red{55}\cdot\red{55}$.

\paragraph{Hint 3}The second term **sixteen times six** is written $\pink{16} \cdot \purple{6}$. We have to **subtract** this term from the first term $\red{55}^{\blue{5}}$.

\paragraph{Hint 4}So we can write **fifty-five raised to the fifth power $\gray{\textrm{minus}}$ sixteen times six** as $\red{55}^{\blue{5}} \gray{-} \pink{16}\cdot \purple{6}$.

\paragraph{Hint 5}If we wanted to evaluate this expression, we must be careful to perform the mathematical operations in the right order. We always compute exponents before multiplications, and multiplications before additions and subtractions:

\begin{align*}
\qquad 
\red{55}^{\blue{5}} \gray{-} \pink{16}\cdot \purple{6}
 &= 55^{\green{5}} - 16\cdot 6 \\
 &= 55\green{\cdot}55\green{\cdot}55\green{\cdot}55\green{\cdot}55 - 16\green{\cdot}6 \\
 &= 503284375 \green{-} 96 \\
 &= 503284279
\end{align*}  

So $503284279$ is also a correct answer since $55^5 - 16\cdot 6 =503284279$.

It is easier to write the short math expression $\red{55}^{\blue{5}} \gray{-} \pink{16}\cdot \purple{6}$ than going through the whole calculation to find the number $503284279$. 



\medskip
\noindent
\textbf{Tags:} {\footnotesize CC.6.EE.A.1, SB.6.1.E.2.CR, Writing expressions 1, CC.6.EE.A.2a, Transcribing expressions}\\
\textbf{Version:} 91dadfb0.. 2013-06-22
\smallskip\hrule





\section{\href{https://www.khanacademy.org/devadmin/content/items/xb0501aa9}{xb0501aa9}}

\noindent
The area of a square is equal to its width times its height. A square with side $x$ has width $\green{x}$ and height $\blue{x}$, so its area is equal to $\green{x} \cdot \blue{x} =x^2$. 

Malina is designing a new logo for her company. She takes a big square of side length $10$ and cuts out a smaller square of side length $3$ in the center to make a giant letter O.

**Write a mathematical expression that describes the area of the logo.** 

\paragraph{Ans}  $10^2 - 3^2$ 

\paragraph{Hint 1}To find the area of the logo, let's first find the area of the big square and then subtract the area of the square that Malina cut out.

\paragraph{Hint 2}We know the area of a square is equal to its side length squared,
so the area of the square with side length $\blue{10}$ is $\blue{10^2}$. 

\paragraph{Hint 3}The area of the little square with side length $\green{3}$, which she cut out from the center, is $\green{3^2}$. 

\paragraph{Hint 4}We can now subtract these two numbers to find the area remaining: $\blue{10^2}-\green{3^2}$. This is the area of the big square that remains after she cut out the piece in the center.




\medskip
\noindent
\textbf{Tags:} {\footnotesize CC.6.EE.A.1, CC.6.EE.A.2c, SB.6.1.E.2.CR, Writing exponential expressions 1, Expressions in word problems}\\
\textbf{Version:} a9a70b23.. 2013-06-22
\smallskip\hrule





\section{\href{https://www.khanacademy.org/devadmin/content/items/xb393b91f}{xb393b91f}}

\noindent
Every weekday, Mario comes home from school at 4 PM and he has to go to bed at 10 PM. He therefore has roughly $6$ hours of free time per weekday. 

Lately he's been watching a lot of TV which leaves him less time for doing stuff like being outdoors and playing with friends. To get an idea of how much time he has left for **doing stuff**, he makes a quick calculation using this table:

free time | time spent watching TV | time for doing stuff
:-: | :-: | :-: 
$6$ | $1$ | $6-1=5$ 
$6$ | $2$ |  $6-2=4$
$6$ | $2.5$ | $6-2.5=3.5$
$6$ | $3$ | $6-3=3$
$6$ | $3.5$ | $6-3.5=2.5$
$6$ | $4$ | $6-4=2$

If the variable $n$ describes how many hours Mario spends watching TV on a given day, **what is the expression which describes how many hours Mario has for doing stuff that day?**



\paragraph{Ans}  $6-n$ 

\paragraph{Hint 1}To find how much time Mario has for doing stuff we must **subtract** the number of hours he spends watching TV from the total number of hours of free time he has after school.

\paragraph{Hint 2}On a usual day he has $\blue{6}$ hours of free time. We said the variable $\red{n}$ represents the number of hours he spends watching TV. What is the number of hours he has left for doing stuff?

\paragraph{Hint 3}The expression which describes the number of hours Mario has for doing stuff is $\blue{6}-\red{n}$.

\paragraph{Hint 4}Looking at this equation Mario realizes something: he can choose $\red{n=0}$ and have a full $\blue{6}$ hours for doing stuff!



\medskip
\noindent
\textbf{Tags:} {\footnotesize SB.6.1.E.2.CR, Writing expressions 1, Abstracting expression from a table, CC.6.EE.A.2}\\
\textbf{Version:} 119bd9a3.. 2013-06-22
\smallskip\hrule





\section{\href{https://www.khanacademy.org/devadmin/content/items/xb3a77977}{xb3a77977}}

\noindent
Enter a mathematical expression which corresponds to ** *x* to the sixth power plus *z* to the seventh power.**

\paragraph{Ans}  $x^6 + z^7$ 

\paragraph{Hint 1}Positive powers on numbers tell us to multiply a number by itself. We are given a verbal expression with two terms with powers. Let's take the terms one by one, and write them out as math.

\paragraph{Hint 2}The first term ** *x* to the sixth power** means we want the number $\red{x}$ multiplied together $\blue{6}$ times: 
$\red{x} \cdot \red{x} \cdot \red{x} \cdot \red{x} \cdot \red{x} \cdot \red{x}  = \red{x}^\blue{6}$.

\paragraph{Hint 3}The second term ** *z* to the seventh power** means we want $\pink{z}$ multiplied together $\purple{7}$ times: $\pink{z} \cdot \pink{z} \cdot \pink{z} \cdot \pink{z} \cdot \pink{z} \cdot \pink{z} \cdot \pink{z} = \pink{z}^\purple{7}$.

\paragraph{Hint 4}So we can write **$\red{x}$ to the $\blue{\textrm{sixth}}$ power plus $\pink{z}$ to the $\purple{\textrm{seventh}}$ power** as 
$\red{x}^\blue{6} + \pink{z}^\purple{7}$. 



\medskip
\noindent
\textbf{Tags:} {\footnotesize SB.6.1.E.2.CR, Writing exponential expressions 1, CC.6.EE.A.2a, Transcribing expressions, CC.6.EE.A.2}\\
\textbf{Version:} c9de90fd.. 2013-06-22
\smallskip\hrule





\section{\href{https://www.khanacademy.org/devadmin/content/items/xbab23e5c}{xbab23e5c}}

\noindent
Write a mathematical expression which corresponds to ** *a* times *b*, subtracted from 42**.

\paragraph{Ans}  $42-ab$ 

\paragraph{Hint 1}We are given a verbal expression which is the **difference** of two terms.  Let's write it out in math.

\paragraph{Hint 2}We start with the quantity ** *a* times *b* ** which is the product of the two variables $\red{a}$ and $\blue{b}$ and is written as $\red{a}\cdot\blue{b}$ or without the dot: $\red{a}\blue{b}$. 

\paragraph{Hint 3}We are asked to **subtract** $\red{a}\blue{b}$ from the number $\green{42}$ so the final expression for ** *a* times *b*, subtracted from 42** is $\green{42}-\red{a}\blue{b}$.



\medskip
\noindent
\textbf{Tags:} {\footnotesize SB.6.1.E.2.CR, Writing expressions 1, CC.6.EE.A.2a, Transcribing expressions, CC.6.EE.A.2}\\
\textbf{Version:} 60d8f006.. 2013-07-16
\smallskip\hrule





\section{\href{https://www.khanacademy.org/devadmin/content/items/xd04429ce}{xd04429ce}}

\noindent
Write a mathematical expression which corresponds to **three times four plus the quantity five times six.**

\paragraph{Ans}  $3*4+5*6$ 

\paragraph{Hint 1}We are given a verbal expression with two terms. 
Let's take the terms one by one.

\paragraph{Hint 2}The first term is **three times four**, which is written $\red{3} \cdot \blue{4}$.

\paragraph{Hint 3}The second term is **five times six**, which is written $\pink{5} \cdot \purple{6}$.

\paragraph{Hint 4}So we can write **three times four plus the quantity five times six** as $\red{3}\cdot\blue{4} + \pink{5}\cdot \purple{6}$.

\paragraph{Hint 5}If you want to evaluate this expression, you compute the multiplications first and then do the addition: 

\begin{align*}
\qquad \red{3}\cdot\blue{4} + \pink{5}\cdot \purple{6} 
 &= 3\green{\cdot}4 + 5\green{\cdot}6  \\
 &= 12 \green{+} 30 \\
 &= 42 
\end{align*}

So $42$ is also a correct answer.



\medskip
\noindent
\textbf{Tags:} {\footnotesize CC.6.EE.A.1, SB.6.1.E.2.CR, Writing expressions 1, CC.6.EE.A.2a, Transcribing expressions}\\
\textbf{Version:} d3f866fb.. 2013-06-22
\smallskip\hrule





\section{\href{https://www.khanacademy.org/devadmin/content/items/xdc8f3e18}{xdc8f3e18}}

\noindent
You have been tasked to prepare the sandwiches for a mountain hike. Each sandwich takes $2$ slices of bread, and you want to prepare $3$ sandwiches for each person going on the hike. **Write an expression which describes how many slices of bread you will need for $n$ people.**

\paragraph{Ans}  $2*3*n$ 

\paragraph{Hint 1}You will need $\pink{2}$ slices of bread per sandwich and  $\blue{3}$ sandwiches per person, which makes $\pink{2}\cdot\blue{3}=6$ slices of bread per person.

\paragraph{Hint 2}If there are $\purple{n}$ people going on the mountain hike, you will need $\pink{2}\cdot\blue{3}\cdot\purple{n}$ slices of bread to prepare the sandwiches.

\paragraph{Hint 3}Note that the expression $\pink{2}\cdot\blue{3}\cdot\purple{n}$ is the same as the expression $6\purple{n}$.



\medskip
\noindent
\textbf{Tags:} {\footnotesize SB.6.1.E.2.CR, Writing expressions 1, Expressions in word problems, CC.6.EE.A.2}\\
\textbf{Version:} 4d0a5b79.. 2013-06-22
\smallskip\hrule



%%  Create a directory called 'figures' in latex dir and run the following command 
%  wget \


\end{document}

